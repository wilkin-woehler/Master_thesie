% !TEX root = writing_version.tex

%\section{Conclusion}
\label{sec:conclusion}
In the thesis at hand the hard sphere nucleation process is studied. The current research topics it thereby addresses are the discrepancy between experimental and theoretical nucleation rates as well as the question if non-Markovian effects are present in the process. An EDMD simulation code that was developed for this purpose during the course of the year is presented together with a thorough testing and extensions for future studies. With the program, large data sets have been produced and characterized in detail.\\

First, we measure the long time self diffusion constant in the metastable hard sphere fluid, as it is necessary to compare the timescale of the system.\\

Secondly, we semi-quantitatively find the duration in which the metastable
clusters in the fluid find their ordering in three different contexts: The initial ordering process after the quench is visible in the cluster size distributions (\autoref{sec:pnt}). Then, in the decay time of the autocovariance functions calculated from the largest cluster in the metastable fluid, we see the correlation time of large fluctuations, probably from distinct clusters (\autoref{sec:acf}). Finally, the duration is seen in direct measurements of the lifetimes of unstable clusters (\autoref{sec:tracking}).\\

Third, the cluster growth and nucleation processes are investigated on. Regarding the growth process, we qualitatively look at shape descriptors derived from the tensor of gyration and try to find correlations between those and different scalar quantities characterizing the growth process. However, we do not find any correlations but still observe that clusters become more and more spherical up to a size of about 1000 particles after which they mostly grow without changing their proportions. The growth takes place with a constant attachment rate to the surface which varies between clusters by about 50\%. Within the narrow interval in which the measurements are done, no dependence on the volume fraction is seen, even though the statistical uncertainties do not exclude a dependence on the diffusion timescale.\\

Regarding the nucleation process, we discussed different definitions of the induction time and quantified the nucleation rate. To do so, we present a combination of the maximum likelihood estimator for censored exponential distributions and a robust Monte Carlo uncertainty estimation which then is employed. The results confirm the discrepancy between experimental and theoretical rates that have been found in previous studies. Nevertheless, the detailed characterization leads us to a possible remedy but it requires further discussion with experimentalists and closer inspection.\\

Last but not least, we present the analysis of a two time memory kernel which has been calculated from trajectories of the largest cluster normalized by the total number of particles. For this purpose, a new ensemble of simulations is produced because the size of the one used to study the nucleation process is too large. The memory kernel on hand includes a Markovian and a non-Markovian contribution. The latter is well described by a Gaussian function of width comparable to the phase transition width of a single trajectory. The extent of the kernel is mostly constant over the whole time range. However, the amplitude increases at first, then remains constant and decreases when most trajectories are in the complete crystalline state.\\
It probably will be necessary to address the question on how the here visible memory contribution depends on the arbitrary box size to reveal possible memory effects originating from the dynamics of the system.

