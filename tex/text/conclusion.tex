% !TEX root = writing_version.tex

%\section{Conclusion}
\label{sec:conclusion}
\newText{
In the thesis at hand the hard sphere nucleation process has been studied. The current research topics it therby tries to address are on the one hand the discrepancy between experimental and theoretical nucleation rates and on the other hand the question if non-Markovian effects are present in the process. For this purpose an EDMD simulation code, that has been developed during the course of the year, is presented together with a thorough testing of it. Furthermore extensions for future studies are discussed. With this program large datasets have been produced and later on characterized in detail.\\
First we measure the long time self diffusion in the metastable hard sphere fluid, as it is necessayr to set the timescale of the system.\\
Secondly we semi-quantitative find the time scale on which the metastable
clusters in the fluid find their ordering and melt in three contexts: The initial ordering process after the quench is visible in the cluster size distribution (\autoref{sec:pnt}), then in the decay time of the autocovariance function calculated from the largest cluster in the metastable fluid (\autoref{sec:acf}) and last in the lifetimes of clusters from direct measurements (\autoref{sec:tracking}).
Third, the cluster nucleation and growth are qunatified. Regarding the growth process we qualitatively look at shape escriptors derived from the tensor of gyration and try to find correlations between those and different scalar quantities characterizing the growth process itself. In this superficial analysis we do not find any correlations, but we can see that cluster become more and more spherical up to a size of about 1000 particles after which they mostly grow without changing their porportions much.\\
The growth itself takes place with a constant attachement rate to the clusters surface. This varies between different clusters by about 50\% but its mean value is not fdepending on the vloume fraction, even though the statistical uncertainties do not exclude a dependence on the diffusion timescale.\\
To quantify the nucleation rates, a combination of the maximum likelihood estimator for censored exponential distributions and a robust Monte Carlo uncertainty estimation are presented and employed. The result confirms the discrepancy between experimental and theoretical rates found in previous studies. Nevertheless the detailed characterization leads us to a possible remedy, requiring further discussions with experimentalists.\\
Last but not least we present the analysis of a two time memory kernel, which has been calculated from trajectories of the largest cluster normalized by the total number of particles. Because the simulation ensemble, used to study the nucleation process is too large for this purpose a more suitable set of simulations is produced. The memory kernel that is found includes a Markovian and a non-Markovian contribution, of which the latter is well described by a Gaussian function. The width of this kernel is found to be comparable to the phase transition width of a single trajectory over the whole time range, while the amplitude increases at first, then remains constant and decreases again when most trajectories are in the complete crystalline state.\\
Addressing the question how the visible memory contribution depence on the arbitrary box size, is probably necessary to reveal memory effects originating from the dynamics of the system.
}
