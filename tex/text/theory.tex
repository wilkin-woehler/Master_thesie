% !TEX root = writing_version.tex

\label{chp:theory}

\section{Hard sphere system}
\label{sec:HS_system}
%Explain the Hard sphere system and the first predictions of the phase transition\\ \\

The Hard Sphere system is the simplest model of a fluid including interactions between the single particles. Its well known potential between particles i and j reads:
\begin{equation}
V(r_{ij})=\infty \cdot \Theta(\sigma - r_{ij})
\end{equation}
In this equation $r_{ij}$ indicates the distance between the two particles, $\sigma$ is the diameter of the Hard Spheres and $\Theta$ is the Heavyside function.\\
While the ideal gas model without pair interactions already makes it possible to derive famous equations as $pV=NkT$, it does not include phase transitions yet. But these can be observed when granting the particles to take up space. Because it is the simplest model and it is well feasible for computer simulations the Hard Sphere system is very well suited to study basic properties of phase transitions.\\ 

Compared to experiments where similar systems can also be realized, general properties of the system at hand can be varied very precisely without much effort and position data of the single particles can be extracted easily as well, because they naturally are required for the simulation.\\

On the downside computer simulations are much more constraint in their size, but with todays computational possibilites system of the order of 1 million particles become tractable, and such computer simulations become a powerfull tool to study also phase transitions in simple systems.\\

The beginning of such simulations actually dates back to the beginning of electronic computer technology (cite Alder and Wainwright 1959). Since then more algorithms to increase efficency have been elaborated, and technology advanced giving today the possibility of studying large systems. 

  
\section{metastable fluid/ phase diagram}
\label{sec:HS_phase_diagram}
The equation of state for the simple Hard Sphere system has various approximations, (cite an overview paper), The probably most common approximation due to its simplicity is the Caranhan-Sterling approximation:
\begin{equation}
Z=\frac{1+\eta+\eta^2-\eta^3}{(1-\eta)^3}
\end{equation}
(cite https://aip.scitation.org/doi/10.1063/1.1672048)
It approximates the compressibility factor Z depending on the packing fraction $\eta$ for the Hard Sphere fluid.\\

For the stable brach after nucleation a common approximation is given by the Alamrza equation of state( cite https://aip.scitation.org/doi/full/10.1063/1.3133328 ).
\begin{equation}
\frac{p(v-v_0)}{k_B T} = 3 - 1.807846 y + 11.56350 y^2 + 141.6 y^3 - 2609.26 y^4 + 19328.09 y^5
\end{equation}
where p is the pressure, v is the volume per particle $v_0=\sigma^3/\sqrt{2}$ is the volume per particle at close packing, including the diameter of the spheres $\sigma$, and $y=p \sigma^3 / (k_B T)$, where $k_B$ is the Boltzman constant and T is the temperature of the crystal.\\

From these two equations of state we can draw the phase diagram:
(include grapic of phase diagram.)\\

The chemical potential difference between the two equations of state can be calculate from the difference between the two equations of state.\\

Eventhough not further discussed in this thesis it might be said that for polydisperse radii the phase diagram becomes even richer as show for example in 
\todo{https://journals.aps.org/prl/pdf/10.1103/PhysRevLett.91.068301}

\section{Classical nucleation theory :/}
\label{sec:CNT}
Classical nucleation theory (CNT) has been proposed \todo{look who did this the first time} and since then multiple times modified to describe various types of systems. Eventhough its predictions often deviate far from experimental results. \todo{look if tanja cited someone else here}. Either way it can provide a reference to know what to expect roughly and also it can be checked by the simulation data.\\

CNT assumes that a spherical crystalite may form in the liquid with properties of the bulk crystal while the fluid remains with the properties of the bulk liquid. The difference in the free energy landscape is given by a surface and a volume term. The first arises from the surface tension $\gamma$ between the fluid bulk and solid bulk phases. The second comes by the difference in chemical potential $\Delta \mu$. The whole expression reads:
\begin{equation}
\beta \Delta G(R) =4 \pi R \gamma -\frac{4}{3} \pi R^3 \rho \Delta \mu  
\end{equation}
Where $\rho$ is the particle density of the solid phase.\\

The free energy barrier can be sketched like this (maybe sketch it for som mu and gamma).\\

As can be seen it has a maximum at $R_{crit}$ the critical radius. If the radius of a cluster surpasses the critical radius, it is likely to continue to grow until it incorporates all available fluid. It is given by:
\begin{equation}
R_{crit} = \frac{2 \gamma}{\rho \Delta \mu }
\end{equation}

Furthermore the height of this barrier can be calculated to be 
\begin{equation}
\beta \Delta G (R_{crit}) = \frac{16 \pi \gamma^3}{3 \rho^2 (\Delta \mu )^2}
\end{equation}

In the classical picture now we look at an activated process which has an rate given by the Arhenius law:
\begin{align}
k&=c \exp{(-\frac{k_B T}{\Delta E})}\\
\Leftrightarrow \quad k &=c \exp{-\beta \Delta G (R_{crit})}
\end{align}
As the constant c is not further defined the absolute nucleation rate in this picture is not predicted but only set into realtion to ther rates. \todo{Check carefully in how far this is correct!}

Given the equation of state for the liquid and fluid phase from \autoref{sec:HS_phase_diagram}, and taking a literature value for $\gamma$ \todo{cite the refernce to which the value corresponds} we can calculate $R_{crit}$ for various densities.

\todo{Plot of $r_{crit}$ for different densities}

\section{Memory including approaches}
\label{sec:memory_approach}
CNT assumes a Markovian system, which means that it diffuses through a free energy landscape without granting it memory. But it has been shown by Kuhnbold ... \todo{cite Anjas Lennard-Jones System} that for a Lennard-Jones system memory effects can not be neglected if an acurate  desciption is desired. Such it must be assumed that also for the Hard Sphere system memory effects may be present.\\

To analyze such memory effects a framework by Hugues Meyer(\todo{cite Hugues}) has been elaborated. In it a memory kernel for coarse-grained observables by means of projection operators is defined together with an efficent algorithm to calculate such.\\
With the derived memory kernel an equation of motion for the observable can be dervied, ressmbling mostly the Generalized Langevin equationbu is called non-stationary Generalized Langevin Equation because the memory kernel can depend not only on $K(t-\tau)$ but instead on two times $K(t,\tau)$:

\begin{equation}
\label{eqn:EOM_A}
  \frac{d A_{t}}{dt} = \omega (t) A_{t} + \int_{0}^{t} d\tau  K(\tau, t) A_{\tau} + \eta_{t} \quad ,
\end{equation}

For the Markovian process the memory kernel is approximately given by a Dirac delta functional $\delta(\tau-t)$, in which case the Langevin equation is recovered.


\section{Computer Precision}
\label{sec:precision}
%Explain a little about what numerics does, compared to the real world.\\
%Maybe include the time evolution of minimal changes
The precision of computer certainly influences the outcome of simulations. (\todo{talk with fabian if he found any papers reagarding varying results fromvarying precision over time})\\
And it should always be kept in mind that the simulation only approximates the real world. Even smallest variations in the last digits of positions, changes the simulation radical after a certain number of steps. This comes by the fact that we face a complex system with chaotic behaviour, which means that even small variations will grow exponentally until the system has nothing in common anymore.  
\todo{Look if you can visulaize such an behaviour by}
As it can be seen after x timesteps the two system have radicallly changed.

\section{Comparsion to Real world experiments} 
\label{sec:comparison}
%Compare, regadring the solvent. Esspeically with Hajos finding.//
%Maybe connect it with the Computer precision part, to talk about what the s<ystem is and what the system is not
Since the \todo{1950s ?} people have systhesized hard sphere like systems in the lab. Today a whole zoo of systems is known. All of these system have in common, that the hard spheres are in a bath of a fluid, which surrounds them. This fluids density and optical refractive index shhould match the density and optical refractive index of the hard spheres to prevent segmentation and to enable optical measurements of the system. \todo{maybe elaborate a little more on why each of the two is necessary}\\

The absence of the bath in simple hard sphere simulations is probably the largest difference to the hard sphere systems in the laboratory. It has been argued that the difference can be circumvented by normalizing with a diffusion length or time, but a discussion on the possibility of hydrondynamic effects changing the behaviour of the lab system compared to simulations is ongoing at the moment.\\ 

For simulations it is much harder to include the bath becuase it introudces a multiple number of particles, making it very slow to simulate large systems.\\




