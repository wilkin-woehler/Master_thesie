% !TEX root = Protokoll.tex
%Encoding
\usepackage[utf8]{inputenc}

%Formatierung Ränder (Optional)
\usepackage[left=2cm,right=2cm,top=2.3cm,bottom=2.3cm,includeheadfoot]{geometry}

%Spracheinstellung
\usepackage[english]{babel}
\usepackage{setspace}
\onehalfspacing
%\addto\captionsngerman{
%	\renewcommand{\figurename}{Abb.}
%	\renewcommand{\tablename}{Tab.}
%}

%Glaettung von Text
\usepackage{lmodern}

%Font-Encoding 
\usepackage[T1]{fontenc} 

%Tabellen und Grafiken
\usepackage{graphicx}
\usepackage{tabularx}
\setlength{\tabcolsep}{18pt}%space between text and left/right border
\usepackage{dcolumn}
\usepackage{multicol}
\usepackage{float}
\usepackage{floatflt}
\usepackage{here}
\usepackage{blindtext}
\newcolumntype{.}{D{.}{.}{7}}
\usepackage{wrapfig}
\usepackage{bigstrut}

\usepackage{subfig}

%\usepackage{subfigure}
\usepackage{here}

%Ueberschrift mit Serifen (nur Inhaltsverzeichnis)
\setkomafont{sectioning}{\normalfont\bfseries}

%Caption
\usepackage{caption}

%Mathe, Physik und Chemiepakete
\usepackage{amsfonts,amsmath,amssymb}
\usepackage[version=3]{mhchem}

%Units/Fraction
\usepackage[output-decimal-marker={.}, per-mode=fraction]{siunitx}
\usepackage{icomma}

%Nummerierung Gleichungen bei mehreren Kapiteln
\numberwithin{equation}{section}
\numberwithin{figure}{section}
\numberwithin{table}{section}

%Sonstiges
\usepackage{geometry}
\geometry{verbose,a4paper,tmargin=25mm,bmargin=25mm,lmargin=15mm,rmargin=20mm}
\usepackage{amsfonts}
\usepackage{amssymb}
\usepackage{latexsym}
\usepackage{texdraw}
\usepackage[T1]{fontenc}
\usepackage[breaklinks,pdfborder={0 0 0}]{hyperref}
\usepackage{url}
\usepackage{pgf}
\usepackage{tikz}
\usetikzlibrary{fit}

%Definierte Wortsilbentrennung
\hyphenation{Test}

%Bilder Ordner
\graphicspath{{plots/}}

%Titel
\title{Dokumenttitel}
\author{Autor}
\usepackage[headsepline]{scrpage2}
\pagestyle{scrheadings}
\clearscrheadfoot
\ihead{Wilkin Woehle}
\ofoot{\pagemark}
\ohead{\headmark}
\automark{section}

%Bibliography
%\usepackage{bibgerm}
\usepackage[backend=biber, style = phys,maxnames=4,clearlang=true,doi=false]{biblatex}
%\DeclareRedundantLanguages{english,german,french,en,de}{english,german,french,en,de}
\addbibresource{bib/library.bib}

\defbibfilter{papers}{
	type=article or
	type=inproceedings or
	type=incollection
} 


%Ueberschriften formatieren
\addtokomafont{title}{\normalfont\bfseries}
\addtokomafont{section}{\normalfont\bfseries\Large}
\addtokomafont{subsection}{\normalfont\bfseries\large}
\addtokomafont{subsubsection}{\normalfont\bfseries\normalsize}
\addtokomafont{paragraph}{\normalfont\bfseries\normalsize}


% Abstand nach math-Umgebungen
\setlength\abovedisplayshortskip{0pt}
\setlength\belowdisplayshortskip{0pt}
\setlength\abovedisplayskip{0pt}
\setlength\belowdisplayskip{0pt}

%Neu
\setlength{\parindent}{0pt}



