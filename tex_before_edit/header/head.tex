% !TEX root = Protokoll.tex
%Encoding
\usepackage[utf8]{inputenc}

%Formatierung Ränder (Optional)
\usepackage[a4paper, left=2.4cm,right=2.4cm,top=2.4cm,bottom=2.4cm,includeheadfoot, bindingoffset=2mm]{geometry}
%\usepackage[twoside, a4paper, bindingoffset=5mm, marginratio={4:6,5:7}, includeheadfoot]{geometry}
%\usepackage{geometry}
%\geometry{verbose,a4paper,tmargin=20mm,bmargin=20mm,lmargin=26mm,rmargin=20mm}
%bmargin etx equivalent to bottom
%inner outer equivalent to left / right

%\usepackage{showframe}

%Spracheinstellung
\usepackage[english]{babel}
\usepackage{setspace}
\onehalfspacing
%\addto\captionsngerman{
%	\renewcommand{\figurename}{Abb.}
%	\renewcommand{\tablename}{Tab.}
%}

%Glaettung von Text
\usepackage{lmodern}

%Font-Encoding 
\usepackage[T1]{fontenc} 

%Tabellen und Grafiken
\usepackage{graphicx}


\usepackage[caption = false]{subfig}
%\usepackage[demo]{graphicx}



%\usepackage{tabularx}
\setlength{\tabcolsep}{18pt}%space between text and left/right border
%\usepackage{dcolumn}
%\usepackage{multicol}

%\usepackage{scrhack}
%\usepackage{float} %Leads to wrong hyperlinks within the document!
%\floatstyle{boxed}
%\restylefloat{figure}
%\usepackage{floatflt}

\usepackage{tocbasic}


%\usepackage{here}
%\usepackage{blindtext}
%\newcolumntype{.}{D{.}{.}{7}}
%\usepackage{wrapfig}
%\usepackage{bigstrut}

%\usepackage{subfig}

%\usepackage{subfigure}
%\usepackage{here}
\usepackage[section]{placeins}

%Ueberschrift mit Serifen (nur Inhaltsverzeichnis)
\setkomafont{sectioning}{\normalfont\bfseries}

%Caption
\usepackage{caption}

%Mathe, Physik und Chemiepakete
\usepackage{amsfonts,amsmath,amssymb}
\usepackage{mathtools}
\usepackage[version=3]{mhchem}

%Units/Fraction
\usepackage[output-decimal-marker={.}, per-mode=fraction]{siunitx}
\usepackage{icomma}

%Nummerierung Gleichungen bei mehreren Kapiteln
\numberwithin{equation}{section}
\numberwithin{figure}{section}
\numberwithin{table}{section}

%Sonstiges
\usepackage{amsfonts}
\usepackage{amssymb}
\usepackage{latexsym}
\usepackage{texdraw}
\usepackage[T1]{fontenc}
\usepackage[breaklinks,pdfborder={0 0 0},plainpages=false,pdfpagelabels]{hyperref}
\usepackage{url}
\usepackage{pgf}
\usepackage{tikz}
\usetikzlibrary{fit}

%Definierte Wortsilbentrennung
\hyphenation{Test}

%Bilder Ordner
\graphicspath{{../plots/}}

%Titel
\title{Dokumenttitel}
\author{Autor}
\usepackage[headsepline]{scrlayer-scrpage}
%\pagestyle{scrheadings}
\clearscrheadfoot


%WEll that took a while...
%\lehead{\headmark}
%\automark{chapter}
%\rehead{Section}

%\lohead{section}

%\automark[chapter]{section}
%\leftmark{section}
%\rightmark{chapter}
%\lehead{\headmark \hfill}
%\lohead{\headmark}
%\rehead{\headmark}
%\rohead{\headmark}
%\cehead{\headmark}
%\cohead{\headmark}

%\automark[chapter]{chapter}

%\rehead{\headmark}
%\automark[chapter]{section}

%\lohead{\headmark}
%\automark[section]{section}

%\lohead{\headmark}
%\automark[section]{section}

%\ohead{Wilkin Wöhler}
%\ihead{\headmark}
\automark[chapter]{section}

\rohead{\leftmark}
%\lehead{Wilkin Wöhler}
\lehead{\rightmark}
\rehead{\leftmark}

\ofoot{\pagemark} %Outerfooter (i-nner, c-enter + all three for head available)

%\ohead{\headmark}
%\automark[section]{section} %\automark[]{} gives as first argument left and second argument right side if two sided document





%Bibliography
%\usepackage{bibgerm}
\usepackage[backend=biber, style = phys,maxnames=4,clearlang=true,doi=false]{biblatex}
\usepackage{csquotes}
%\DeclareRedundantLanguages{english,german,french,en,de}{english,german,french,en,de}
\addbibresource{bibliography/library.bib}

\defbibfilter{papers}{
	type=article or
	type=inproceedings or
	type=incollection
} 


%Ueberschriften formatieren
\addtokomafont{title}{\normalfont\bfseries}
\addtokomafont{section}{\normalfont\bfseries\Large}
\addtokomafont{subsection}{\normalfont\bfseries\large}
\addtokomafont{subsubsection}{\normalfont\bfseries\normalsize}
\addtokomafont{paragraph}{\normalfont\bfseries\normalsize}


% Abstand nach math-Umgebungen
\setlength\abovedisplayshortskip{0pt}
\setlength\belowdisplayshortskip{0pt}
\setlength\abovedisplayskip{0pt}
\setlength\belowdisplayskip{0pt}

%Neu
\setlength{\parindent}{0pt}



