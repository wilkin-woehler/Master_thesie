\documentclass[a4paper, twoside, 11pt]{scrreprt}

% !TEX root = Protokoll.tex
%Encoding
\usepackage[utf8]{inputenc}

%Formatierung Ränder (Optional)
\usepackage[left=2cm,right=2cm,top=2.3cm,bottom=2.3cm,includeheadfoot]{geometry}
%\usepackage{showframe}

%Spracheinstellung
\usepackage[english]{babel}
\usepackage{setspace}
\onehalfspacing
%\addto\captionsngerman{
%	\renewcommand{\figurename}{Abb.}
%	\renewcommand{\tablename}{Tab.}
%}

%Glaettung von Text
\usepackage{lmodern}

%Font-Encoding 
\usepackage[T1]{fontenc} 

%Tabellen und Grafiken
\usepackage{graphicx}


\usepackage[caption = false]{subfig}
%\usepackage[demo]{graphicx}



%\usepackage{tabularx}
\setlength{\tabcolsep}{18pt}%space between text and left/right border
%\usepackage{dcolumn}
%\usepackage{multicol}
\usepackage{float}
%\floatstyle{boxed}
\restylefloat{figure}
%\usepackage{floatflt}
%\usepackage{here}
%\usepackage{blindtext}
%\newcolumntype{.}{D{.}{.}{7}}
%\usepackage{wrapfig}
%\usepackage{bigstrut}

%\usepackage{subfig}

%\usepackage{subfigure}
%\usepackage{here}
\usepackage[section]{placeins}

%Ueberschrift mit Serifen (nur Inhaltsverzeichnis)
\setkomafont{sectioning}{\normalfont\bfseries}

%Caption
\usepackage{caption}

%Mathe, Physik und Chemiepakete
\usepackage{amsfonts,amsmath,amssymb}
\usepackage{mathtools}
\usepackage[version=3]{mhchem}

%Units/Fraction
\usepackage[output-decimal-marker={.}, per-mode=fraction]{siunitx}
\usepackage{icomma}

%Nummerierung Gleichungen bei mehreren Kapiteln
\numberwithin{equation}{section}
\numberwithin{figure}{section}
\numberwithin{table}{section}

%Sonstiges
\usepackage{geometry}
\geometry{verbose,a4paper,tmargin=25mm,bmargin=25mm,lmargin=20mm,rmargin=20mm}
\usepackage{amsfonts}
\usepackage{amssymb}
\usepackage{latexsym}
\usepackage{texdraw}
\usepackage[T1]{fontenc}
\usepackage[breaklinks,pdfborder={0 0 0}]{hyperref}
\usepackage{url}
\usepackage{pgf}
\usepackage{tikz}
\usetikzlibrary{fit}

%Definierte Wortsilbentrennung
\hyphenation{Test}

%Bilder Ordner
\graphicspath{{../plots/}}

%Titel
\title{Dokumenttitel}
\author{Autor}
\usepackage[headsepline]{scrpage2}
\pagestyle{scrheadings}
\clearscrheadfoot


%WEll that took a while...
%\lehead{\headmark}
%\automark{chapter}
%\rehead{Section}

%\lohead{section}

%\automark[chapter]{section}
%\leftmark{section}
%\rightmark{chapter}
%\lehead{\headmark \hfill}
%\lohead{\headmark}
%\rehead{\headmark}
%\rohead{\headmark}
%\cehead{\headmark}
%\cohead{\headmark}

%\automark[chapter]{chapter}

%\rehead{\headmark}
%\automark[chapter]{section}

%\lohead{\headmark}
%\automark[section]{section}

%\lohead{\headmark}
%\automark[section]{section}

%\ohead{Wilkin Wöhler}
%\ihead{\headmark}
\automark[chapter]{section}

\rehead{\leftmark}
%\lehead{Wilkin Wöhler}
\lohead{\rightmark}
\rohead{\leftmark}

\ofoot{\pagemark} %Outerfooter (i-nner, c-enter + all three for head available)

%\ohead{\headmark}
%\automark[section]{section} %\automark[]{} gives as first argument left and second argument right side if two sided document





%Bibliography
%\usepackage{bibgerm}
\usepackage[backend=biber, style = phys,maxnames=4,clearlang=true,doi=false]{biblatex}
%\DeclareRedundantLanguages{english,german,french,en,de}{english,german,french,en,de}
\addbibresource{bibliography/library.bib}

\defbibfilter{papers}{
	type=article or
	type=inproceedings or
	type=incollection
} 


%Ueberschriften formatieren
\addtokomafont{title}{\normalfont\bfseries}
\addtokomafont{section}{\normalfont\bfseries\Large}
\addtokomafont{subsection}{\normalfont\bfseries\large}
\addtokomafont{subsubsection}{\normalfont\bfseries\normalsize}
\addtokomafont{paragraph}{\normalfont\bfseries\normalsize}


% Abstand nach math-Umgebungen
\setlength\abovedisplayshortskip{0pt}
\setlength\belowdisplayshortskip{0pt}
\setlength\abovedisplayskip{0pt}
\setlength\belowdisplayskip{0pt}

%Neu
\setlength{\parindent}{0pt}




% !TEX root = Protokoll.tex


%Namen Physiker
\newcommand{\Snellius}{\textnormal{\textsc{Snellius}} }
\newcommand{\Bohr}{\textnormal{\textsc{Bohr}} }
\newcommand{\Pickering}{\textnormal{\textsc{Pickering}} }
\newcommand{\Balmer}{\textnormal{\textsc{Balmer}} }
\newcommand{\Planck}{\textnormal{\textsc{Planck}} }
\newcommand{\Einstein}{\textnormal{\textsc{Einstein}} }
\newcommand{\Rydberg}{\textnormal{\textsc{Rydberg}} }
\newcommand{\Gauss}{\textnormal{\textsc{Gauß}} }
\newcommand{\Heisenberg}{\textnormal{\textsc{Heisenberg}} }
\newcommand{\Parseval}{\textnormal{\textsc{Parseval}} }
\newcommand{\Schroedinger}{\textnormal{\textsc{Schrödinger}} }
\newcommand{\Hamilton}{\textnormal{\textsc{Hamilton}} }
\newcommand{\Doppler}{\textnormal{\textsc{Doppler}} }
\newcommand{\Zeeman}{\textnormal{\textsc{Zeeman}} }
\newcommand{\FabryPerot}{\textnormal{\textsc{Fabry-Pérot}} }
\newcommand{\BiotSavart}{\textnormal{\textsc{Biot-Savart}} }
\newcommand{\Lande}{\textnormal{\textsc{Landé}} }
\newcommand{\Lorentz}{\textnormal{\textsc{Lorentz}} }
\newcommand{\Maxwell}{\textnormal{\textsc{Maxwell}} }
\newcommand{\CG}{\textnormal{\textsc{Clebsch-Gordon}} }
\newcommand{\Tesla}{\textnormal{\textsc{Tesla}} }


%Neue Komandos:
\newcommand{\todo}[1]{\textcolor{red}{#1}}
\newcommand{\al}[1]{\begin{align}
	#1
	\end{align}}
\newcommand{\ali}[1]{\begin{align*}
	#1
	\end{align*}}
\newcommand{\inv}[1]{\frac{1}{#1}}
\newcommand{\itbf}[1]{\textit{\textbf{#1}}}
\renewcommand{\it}[1]{\textit{#1}}
\renewcommand{\sc}[1]{\textsc{#1}}
\renewcommand{\d}{\mathrm{d}}


\makeatletter
     \renewcommand*\l@figure{\@dottedtocline{1}{1em}{3.0em}}
     \renewcommand*\l@table{\@dottedtocline{1}{1em}{3.2em}}
\makeatother



%88888888888888888888888888888888888888888888888888888888888888888888888888
\begin{document}
	%set language
	\selectlanguage{english}
	
	%sections
	\renewcommand{\subsectionautorefname}{section\negthinspace}
	\renewcommand{\subsubsectionautorefname}{section\negthinspace}	\renewcommand{\figureautorefname}{fig.\negthinspace}
	\renewcommand{\tableautorefname}{tab.\negthinspace}
	\renewcommand{\equationautorefname}{eq.\negthinspace}
	
	
	\newcommand{\vtitle}{Nucleation and Crystallization of the Metastable Hard Sphere Fluid}
	\newcommand{\vinstitute}{Research group for complex systems and soft matter}
	\newcommand{\vsupervision}{Prof. Dr. Tanja Schilling}
	\newcommand{\vauthor}{Wilkin Wöhler}

	
	
	\newcommand{\presummary}{Nucleation and cluster development in the metastable hard sphere fluid are studied in this thesis. To this purpose an event driven molecular dynamcis simulation code is developed and thouroughly tested by measuring well known quantities like diffusion coefficients or radial distribution functions at various packing fractions. Its performance is well suited for large systems enabling also the measurement of cluster growth rates and shape descriptors for clusters with sizes up to a hundred thousand particles without significant spatial influence to itself due to the periodic boundary conditions.\\ 
The program is used to simulate 2000 trajectories of systems containing about 1 million particles. In the analysis of the data a constant attachment rate to the cluster's surface is measured. Surprisingly the attachement rate seems unaffected by the packing fraction of the surrounding mestastable liquid, even though varying between single clusters by about 50\% leading to uncertainties that can not exclude a dependence on the diffusion time scale.\\
From shape descriptors based on the Tensor of Gyration a tendency towards more spherical clusters is observed up to sizes of about a thousand particles. Clusters including more particles seem to conserve their almost spherical proportions and approach the completly spherical shape only slowly.\\  
Also nucleation rates at volume fractions of $\eta \in [53.1\%,53.4\%]$ are measured at high precision compared to earlier measurements of these, confirming the discrepancy between real world experiments and numerical simulations. Beyond that the impact memory effects in nucleations for smaller systems is investigated with the finding of a Gaussian memory kernel. The width of which therby is comparable to the width of the phase transition time for a single trajectory, as previously shown by Meyer et al. 2021\cite{Meyer2021}.}

\pagenumbering{Roman} % and numbering to roman numbering
%\pagestyle{plain} % Set the pagestyle to plain to exclude header in inital part

% !TEX root = Protokoll.tex
\newcommand{\HRule}{\rule{\linewidth}{0.2mm}}

\begin{titlepage}	
	\begin{center}
		%\mbox{}
		%\vspace{-10pt}

		\HRule \\[0.8cm]
		{ \huge \bfseries \vtitle}\\[0.4cm]
		\HRule \\[0.4cm]

		\vspace{10pt}

		{ \Large by \vauthor }

		\vspace{120pt}
		
		\textsc{\Large Physics - Master Thesis\\[0.5cm] 
			at the Albert-Ludwigs University of Freiburg\\[0.5cm]
			May 2021\\[0.5cm]}

		\vspace{100pt}	

		\Large{	Elaborated within the\\ 
			\vinstitute\\
			supervised by \\
 			\vsupervision\\}

	\end{center}
	\vspace{2cm}
\end{titlepage}
\pagebreak

%\null\thispagestyle{empty}  %An empty page
%\newpage


%\raggedright
\hspace{2cm}

{\large Ich versichere, dass ich die Arbeit selbstständig verfasst und keine anderen als die 		angegebenen Quellen und Hilfsmittel benutzt, sowie Zitate kenntlich gemacht habe.\par} 
\vspace{1.2cm}
{\large Freiburg, den } \underline{\hspace{0.6cm}} . \underline{\hspace{0.6cm}} . \underline{\hspace{1.2cm}}   \hspace{2.5cm} \underline{\hspace{3.5cm}}
\newpage


\mbox{}
\vspace{10pt}
\begin{center}
	\textbf{Abstract}\qquad
\end{center}
	\presummary




\renewcommand{\chapterpagestyle}{plain}

\tableofcontents 
\newpage

\listoffigures
\newpage

\listoftables
\newpage

\null\thispagestyle{empty}  %An empty page
\newpage


\pagenumbering{arabic} % Reset the numbering to arabic numbering
\pagestyle{scrheadings} % Set the pagestyle back to header including style

%\renewcommand{\chapterpagestyle}{scrheadings} % Set the pagestyle of chapter page also with header


\chapter{Introduction}
% !TEX root = writing_version.tex

\label{chp:theory}

\section{Hard sphere system}
\label{sec:HS_system}
The hard sphere system is the simplest model of a fluid going beyond the ideal gas only by including interactions between the single particles in the form of an occupied volume. Its well known potential between particles i and j is given in \autoref{eqn:hs_potential}.
\begin{equation}
\label{eqn:hs_potential}
V(r_{ij})=%\infty \cdot \Theta(\sigma - r_{ij})
\begin{cases}
\infty \quad & r_{ij} \le \sigma \\
0 \quad & r_{ij} > \sigma
\end{cases}
\end{equation}
In this equation $r_{ij} = r_j - r_i$ names the distance between the two particles and $\sigma$ is the diameter of the hard spheres.\\

While the ideal gas model without pair interactions already makes it possible to derive the famous equation of state $pV=NkT$, it does not include phase transitions yet. But these can be observed when granting the particles to occupy space in the simplest case by defining hard spheres of the kind in \autoref{eqn:hs_potential}. Because it is the simplest model and it is well feasible for computer simulations the hard sphere system is very well suited to study basic properties of first order phase transitions.\\ 

Compared to experiments where similar systems are possible, general properties of the system at hand can be varied very precisely without much effort and information about each single particle can be extracted easily as they are naturally required for the simulation.\\

On the downside computer simulations are much more constraint in their size, but with today's computational possibilities system of the order of 1 million particles become tractable, and such computer simulations become a powerful tool to study also phase transitions in simple systems.\\

The beginning of such simulations dates back to the beginning of electronic computer technology with first studies by Alder and Wainwright in 1959 \todo{cit the paper}. Since then more algorithms to increase efficiency have been elaborated, and technology advanced giving today the possibility of studying large scale systems. 

\section{The phase diagram and the meta stable fluid}
\label{sec:HS_phase_diagram}
The equation of state for the monodisperse hard sphere system has various approximations, \todo{cite an overview paper}. The most common approximation due to its simplicity is the Carnahan-Starling approximation
\begin{equation}
\label{eqn:CS}
Z=\frac{1+\eta+\eta^2-\eta^3}{(1-\eta)^3} \; \text{.}
\end{equation}
\todo{cite https://aip.scitation.org/doi/10.1063/1.1672048}
It approximates the compressibility factor Z depending on the packing fraction $\eta$ for the hard sphere fluid.\\

For the stable solid branch a common approximation is given by the Almarza equation of state \todo{cite https://aip.scitation.org/doi/full/10.1063/1.3133328}
\begin{equation}
\frac{p(v-v_0)}{k_B T} = 3 - 1.807846 y + 11.56350 y^2 + 141.6 y^3 - 2609.26 y^4 + 19328.09 y^5 \; \text{.}
\end{equation}
where p is the pressure, v is the volume per particle $v_0=\sigma^3/\sqrt{2}$ is the volume per particle at close packing, including the diameter of the spheres $\sigma$, and $y=p \sigma^3 / (k_B T)$, where $k_B$ is the Boltzmann constant and T is the temperature of the crystal.\\
Here the inverse of the volume per particle is the number of particles per volume $\rho = v^{-1}$. The relation to the corresponding packing fraction $\eta$ is given by $\rho = \frac{6}{ \pi} \eta$, which can be easily shown by extending $\rho = \frac{N}{V}$ by a single particles volume $V_s = \frac{4}{3} \pi \left(\frac{\sigma}{2}\right)^3 = \frac{\pi}{6} \sigma^3$.\\
Within the thesis mostly the volume fraction is used as it is the most common parameter for describing the system. Also the sphere diameter $\sigma$ is used as unit of length.\\ 

Returning to the phase diagram, the volume fraction for freezing has been determined to $\eta_{freeze} = 0.494$, the melting volume fraction has been calculated to be $\eta_{melt}=0.55$. In between these two both phases are in coexistence. This can be understood in the following way: The liquid may follow its branch to pressures above the coexistence pressure. But then it becomes more favourable for the particles to arrange into the crystalline phase as each single particle can access a larger free volume in the structured lattice than it would be possible in the unordered fluid.\\
By comparing the volume fractions of random close packing $\eta_{RCP}\approx 64\%$ with the one of a face centered cubic or hexagonal close packing fraction of $\eta_{HCP} \approx 74 \%$ this becomes clear. Within the crystalline phase each particle still has free volume accessible while the randomly packed particles are already confined at exactly one place.\\
The extra accessible volume translates into a larger number of possible states for the particle or in terms of thermodynamics a larger entropy, and such the metastable fluid will eventually find its equilibrium in the solid phase. As the solid phase will form with a volume fraction corresponding to $\eta_{melt}=0.55$ which is higher than the one of the metastable fluid, the pressure is mitigated and such not all fluid transforms into the solid phase, but both phases may coexist.\\
The overall phase diagram is shown with the constant coexistence pressure in \autoref{fig:hs_phase_diagram}.\\
\begin{figure}[h]
\centering
\includegraphics[width=0.6 \linewidth]{Hard_sphere_phase_diagram.pdf}
\caption{Phase diagram of the hard sphere system with freezing and melting volume fraction shown as shaded lines and the green dashed line indicating the equilibrium stable branch. Where liquid and solid branch do not coincide with the stable branch these are unstable and tend towards the stable branch.}
\label{fig:hs_phase_diagram}
\end{figure}





The solid fraction in terms of volume for the system $x_s = \frac{V_s}{V}$ with $V_s$ the solid volume and $V$ the total volume can be described within the coexistence region in the equilibrium state by \autoref{eqn:solid_fraction}.\\ 
For the derivation it is necessary to use that in the equilibrium state the density of the solid phase is given by the melting density and that the liquid density is equal to the freezing density, i.e $\rho_s = \rho_{melt}$ and $\rho_l = \rho_{freeze}$ respectively.\\
When further using the rather trivial equations
\begin{align}
V &= V_s + V_l \; \text{,} \nonumber\\
N &= n_s + n_l \; \text{,} \nonumber\\
N_i &= \rho_i V_i \; \text{,} 
\end{align}
with $n_{s/l}$ the number of solid/liquid particles we may write:
\begin{align}
\label{eqn:solid_fraction}
\rho V &= \rho_s V_s + \rho_l V_l \nonumber\\
\stackrel{\text{equil.}}{\Leftrightarrow} \quad \rho &= \rho_{melt} \frac{V_s}{V} + \rho_{freeze} \frac{V_l}{V} \nonumber \\
\Leftrightarrow \quad \rho &= \rho_{melt} \frac{V_s}{V} + \rho_{freeze} \frac{V - V_s}{V} \nonumber \\
\Leftrightarrow \rho &= \rho_{melt} \frac{V_s}{V} + \rho_{freeze} \left( 1- \frac{V_s}{V} \right) \nonumber \\
\Leftrightarrow \frac{V_s}{V} &= \frac{\rho - \rho_{freeze}}{\rho_{melt} - \rho_{freeze} } 
\end{align}
As the solid fraction below $\rho_{freeze} $ vanishes and above $\rho_{melt}$ is 1, we can conclude that the equilibrium solid fraction of the system is given by \autoref{eqn:solid_fraction_result}.
\begin{align}
\label{eqn:solid_fraction_result}
x_s(\rho) = 
\begin{cases}
0 & \rho <  \rho_{freeze}\\
\frac{\rho-\rho_{freeze}}{\rho_{melt}-\rho_{freeze}} &  \rho_{freeze} < \rho <  \rho_{melt}\\ 
1 &  \rho > \rho_{melt} \quad \quad \text{.}
\end{cases}
\end{align}

%\begin{align}
%\label{eqn:solid_fraction}
%N \rho &= n_s \rho_s + n_l \rho_l \nonumber\\
%\Leftrightarrow \quad \rho &= \frac{n_s}{N} \rho_s + \frac{N-n_s}{N} \rho_l \nonumber\\
%\Leftrightarrow \quad \rho &= x_s \rho_s + (1-x_s) \rho_l \nonumber\\
%\Leftrightarrow \quad x_s &= \frac{\rho-\rho_l}{\rho_s-\rho_l} \; \text{,}
%\end{align} 
%where further $n_l = N - n_s$ is the number of liquid particles. As the solid fraction below $%\eta_{freeze} $ vanishes and above $\eta_{solid}$ is 1, we can conclude that the equilibrium solid %fraction of the system is given by \autoref{eqn:solid_fraction_result}.
%\begin{align}
%\label{eqn:solid_fraction_result}
%x_s(\rho) = 
%\begin{cases}
%0 & \rho <  \rho_l\\
%\frac{\rho-\rho_l}{\rho_s-\rho_l} &  \rho_l < \rho <  \rho_s\\ 
%1 &  \rho > \rho_s \quad \quad \text{.}
%\end{cases}
%\end{align}


Evaluating the above result at feasible volume fractions for nucleation in between $\eta \in [0.53,0.55]$ leads to coexistence fractions of $x_s \in [0.7,1]$. This means that we are expecting large parts of the system to be in the solid phase after long waiting times.\\

As pointed out earlier the phase transition takes place as it reduces the pressure in the liquid. This means that already during the growth of clusters the volume fraction of the metastable liquid is reduced, potentially altering its behaviour significantly. For closer inspection of this the particle density of the metastable liquid depending on the solid fraction $x_s$ is evaluated in \autoref{eqn:meta_stable_volume_fraction}. For this purpose first the liquid volume $V_l$ and the number of liquid particles $N_l$ is expressed in terms of the solid fraction $x_s$:
\begin{align}
V_l(x_s) & = V(1-x_s)\\
N_l(x_s) & = N-n_s(x_s) = N - \rho_m V x_s = N(1-\frac{\rho_m}{\rho}x_s)
\end{align}
Following it is combined with the result:
\begin{align}
\label{eqn:meta_stable_volume_fraction}
\rho_l(x_s) &= \frac{N_l (x_s) }{ V_l(x_s) } = \frac{N}{V} \frac{1-\frac{\rho_m}{\rho}x_s}{1-x_s} = \rho \frac{1-\frac{\rho_m}{\rho}x_s}{1-x_s}
\end{align}
Some examples of \autoref{eqn:meta_stable_volume_fraction} are depicted in \autoref{fig:remaining_density} for moderate solid fractions of the system at regular volume fractions used for nucleation.\\
\begin{figure}[h]
\centering
\includegraphics[width=0.7 \linewidth]{remaining_density.pdf}
\caption{Visualization of \autoref{eqn:meta_stable_volume_fraction}. The volume fraction of the remaining liquid decreases for all shown initial volume fractions only little up to crystalline ratios of a few percent.}
\label{fig:remaining_density}
\end{figure}
What can be seen is that for crystalline fractions of a few percent the remaining liquid is not altered significantly. Especially for system sizes of about 1 million particles it corresponds to cluster sizes of a few ten thousand particles, where stable growth of clusters takes place which is rather insensitive to changes of the volume fraction as shown in \autoref{sec:cluster_growth}. This means that during the highly sensitive cluster forming processes the volume fraction of the liquid can be assumed to be stable.\\ 

%Eventhough not further discussed in this thesis it might be said that for polydisperse radii the phase diagram becomes even richer as show for example in 
%\todo{https://journals.aps.org/prl/pdf/10.1103/PhysRevLett.91.068301}

\section{Classical nucleation theory }
\label{sec:CNT}
Classical nucleation theory (CNT) has been proposed \todo{look who did this the first time} and since then multiple times modified to describe various types of systems to account for deviations off experimental results \todo{look if tanja cited someone else here}. Still it provides some reference  or expectation to compare with the simulation data.\\

CNT assumes that a spherical crystallite may form in the liquid with properties of the bulk crystal while the fluid remains with the properties of the bulk liquid. The difference in the free energy landscape is given by a surface and a volume term. The first arises from the surface tension $\gamma$ between the fluid bulk and solid bulk phases. The second comes by the difference in chemical potential $\Delta \mu$. The whole expression reads:
\begin{equation}
\label{eqn:free_energy}
\beta \Delta G(R) =4 \pi R \gamma -\frac{4}{3} \pi R^3 \rho \Delta \mu  
\end{equation}
Where $\rho$ is the particle density of the solid phase.\\

For the difference of the chemical potential $\Delta \mu $ we use the free energy difference between the metastable liquid branch and the stable coexistence branch. To calculate it we employ the differential relation of the free energy
\begin{align}
\label{eqn:differential_relation}
dF = -S  \, dT -P \, dV + \mu  \, dN
\end{align}
at a constant number of particles and constant temperature. By further reformulating $dV$ by using $ dN = dV  \, \rho + V  \, d\rho  $ and $dN = 0 $ we find  $ dV = -d\rho \frac{N}{\rho^2}$. With it \autoref{eqn:differential_relation} becomes
\begin{align}
\label{eqn:df_relation}
\frac{dF}{N} = \frac{P(\rho)}{\rho^2} d\rho \; \text{.}
\end{align}

The pressure $p(\rho)$ is given by the equation of state and approximated by the Carnahan-Starling approximation \autoref{eqn:CS} where $\eta = \frac{6 \rho }{\pi}$ and $Z=\frac{pV}{NkT} = \frac{p(\rho)}{\rho kT}$. Integrating \autoref{eqn:df_relation} between two densities leads to
\begin{equation}
\frac{\Delta F}{N} = \int_{\rho_1}^{\rho^2} \frac{kT}{\rho} \frac{1+\left( \frac{6 \rho}{\pi}\right) +\left( \frac{6 \rho}{\pi}\right)^2 - \left( \frac{6 \rho}{\pi}\right)^3}{\left( 1 - \frac{6 \rho}{\pi}\right)^3} d\rho
\end{equation}
which has the analytical solution:
\begin{equation}
\int_{x_1}^{x_2} \frac{1+(ax) +(ax)^2 - (ax)^3 }{( 1 - ax )^3 x} dx = \left. \frac{3-2ax}{(ax-1)^2} + \text{log}(x) \right|_{x=x_1}^{x_2}
\end{equation}
further omitting the lengthy notation for $\eta = \left( \frac{6 \rho}{\pi}\right)$ we end up with
\begin{equation}
\frac{\Delta F}{N} = kT \left(  \frac{3-2 \eta_2}{(\eta_2 - 1)^2} - \frac{3-2 \eta_1}{(\eta_1 - 1)^2} + \text{log}\left( \frac{\eta_2}{\eta_1} \right) \right)
\end{equation}
The analytical solution is compared in \autoref{fig:free_energy_diff} with numerically found results which have been calculated before the analytical solution was found. The difference in free energy is in the following identified with the difference in chemical potential $\Delta \mu$. \\

\begin{figure}[h]
\centering
\includegraphics[width=0.7 \linewidth]{Free_energy_difference.pdf}
\caption{Free energy difference per particle between the metastable liquid phase and the coexistence phase. Values found in the literature deviate from the shown result, but it is speculated that a factor of $\frac{\pi}{6}$ common in the calculations is missing in either this or their calculation, as the modified green curve collapses rather accurately on the literature values when choosing $\eta_{freeze}=0.5$.}
\label{fig:free_energy_diff}
\end{figure}


Coming back to the free energy landscape of \autoref{eqn:free_energy} we see that it exhibits a maximum at a radius called $R_{crit}$. The interpretation of this radius is that if a cluster surpasses the critical radius it is likely to continue to grow until it incorporates all available fluid. Cluster in this sense is defined as a structure having a crystalline like ordering locally. The critical radius is given by \autoref{eqn:r_crit}.

\begin{equation}
\label{eqn:r_crit}
R_{crit} = \frac{2 \gamma}{\rho \Delta \mu }
\end{equation}

Furthermore the height of this barrier can be calculated by common algebra to be 
\begin{equation}
\beta \Delta G (R_{crit}) = \frac{16 \pi \gamma^3}{3 \rho^2 (\Delta \mu )^2}
\end{equation}

The classical calculated critical radius depending on the volume fraction is depicted in \autoref{fig:r_crit} for an first impression of the cluster sizes that we are expecting for nucleation. The value of the surface tension is set to $\gamma = 0.6$. \todo{fix value and cite} 
\begin{figure}[h]
\centering
\includegraphics[width=0.7 \linewidth]{CNT_radius.pdf}
\caption{Critical radius $R_{crit}$ calculated from CNT depending on volume fraction $\eta$. As it can be seen the critical radii are rather small. When using the chemical potential calculated by Fillion and Schilling the critical clusters sizes become of the order $N \approx 50$ at intermediate metastable volume fractions, which is much more in agreement with typical largest cluster fluctuations found in simulations. \todo{maybe something is wrong after all with the derivation ?}}
\label{fig:r_crit}
\end{figure}




%To continue in the classical picture an activated process like this is described by a rate given by the Arhenius law:
%\begin{align}
%k&=c \exp{(-\frac{k_B T}{\Delta E})}\\
%\Leftrightarrow \quad k &=c \exp{-\beta \Delta G (R_{crit})}
%\text{\todo{THIS HAS TO BE CORECCTED}}
%\end{align}
%As the constant c is not further defined the absolute nucleation rate in this picture is not predicted but only set into realtion to ther rates. \todo{Check carefully in how far this is correct!}

\section{Memory including approaches}
\label{sec:memory_approach}
CNT assumes a Markovian system, which means that it diffuses through a free energy landscape without granting it memory. But it has been shown by Kuhnbold ... \todo{cite Anjas Lennard-Jones System} that for a Lennard-Jones system memory effects can not be neglected if an accurate  description is desired. Such it must be assumed that also for the Hard Sphere system memory effects may be present.\\

To analyze such memory effects a framework by Hugues Meyer(\todo{cite Hugues}) has been elaborated. In it a memory kernel for coarse-grained observables by means of projection operators is defined together with an efficient algorithm to calculate such.\\
With the derived memory kernel an equation of motion for the observable can be derived, resembling mostly the Generalized Langevin equation is called non-stationary Generalized Langevin Equation because the memory kernel can depend not only on $K(t-\tau)$ but instead on two times $K(t,\tau)$:

\begin{equation}
\label{eqn:EOM_A}
  \frac{d A_{t}}{dt} = \omega (t) A_{t} + \int_{0}^{t} d\tau  K(\tau, t) A_{\tau} + \eta_{t} \quad ,
\end{equation}

For the Markovian process the memory kernel is approximately given by a Dirac delta functional $\delta(\tau-t)$, in which case the Langevin equation is recovered.


\section{Computer Precision}
\label{sec:precision}
%Explain a little about what numerics does, compared to the real world.\\
%Maybe include the time evolution of minimal changes
The precision of computer certainly influences the outcome of simulations. (\todo{talk with Fabian if he found any papers regarding varying results from varying precision over time})\\
And it should always be kept in mind that the simulation only approximates the real world. Even smallest variations in the last digits of positions, changes the simulation radical after a certain number of steps. This comes by the fact that we face a complex system with chaotic behaviour, which means that even small variations will grow exponentially until the system has nothing in common anymore.  
\todo{Look if you can visualize such an behaviour by}
As it can be seen after x timesteps the two system have radically changed.

\section{Comparison to Real world experiments} 
\label{sec:comparison}
%Compare, regadring the solvent. Esspeically with Hajos finding.//
%Maybe connect it with the Computer precision part, to talk about what the s<ystem is and what the system is not
Since the \todo{1950s ?} people have synthesized hard sphere like systems in the lab. Today a whole zoo of systems is known. All of these system have in common, that the hard spheres are in a bath of a fluid, which surrounds them. This fluids density and optical refractive index should match the density and optical refractive index of the hard spheres to prevent segmentation and to enable optical measurements of the system. \todo{maybe elaborate a little more on why each of the two is necessary}\\

The absence of the bath in simple hard sphere simulations is probably the largest difference to the hard sphere systems in the laboratory. It has been argued that the difference can be circumvented by normalizing with a diffusion length or time, but a discussion on the possibility of hydrodynamic effects changing the behaviour of the lab system compared to simulations is ongoing at the moment.\\ 

For simulations it is much harder to include the bath because it introduces a multiple number of particles, making it very slow to simulate large systems.\\





\newpage

\chapter{Simulation details}
% !TEX root = writing_version.tex

\label{chp:simulation}
During the coarse of the master thesis an event driven molecular dynamic (EDMD) simulation code has been elaborated. The choice to use the EDMD approach is taken because interst in the actual dynamics of the system were desired. This means that simulations probing the phase space of the system instead of the dynamics, like Monte Carlo (MC) simulation schemes, are not suited.\\ 
Furthermore the discontinous potential of the hard spheres is an obstacle not easy to face in regular molecular dynamics (MD) schemes, where the Newtonian equation of motion for the particles is numerically integrated.\\
The EDMD approach on the other site actually requires these discontinuities as will be discussed in the following sections, together with some details of the program.\\

\section{Algorithm and Simulation details}
\label{sec:simulation}
%EDMD and Simulation details
In this section we will highlight the main differences to regular MD simulations, as they are the main tool to otherwise probe the dynamics of the system. Furthermore we will stick to the hard sphere example when discussing the EDMD simulations, but it can be kept in mind that the EDMD approach also allows to simulate particles with other potentials as long as the potentials are only containing step functions.\\
 
The decisive difference between EDMD simulations and regular MD schemesis that, instead of evaluating all pair and external forces on each particle and then evolving the whole system to the next time step, EDMD simulations do not have a predefined time step, but the system is evolved from one event to the next one. An event in this context is defined as the time where the next collision in the whole system takes place.\\

The event prediction algorithm is follows closely the approach proposed by Bannerman et. al \cite{Bannerman2014} which will be discussed in the next section.\\

\subsection{Event driven molecular dynamics (EDMD)}
\label{sec:EDMD}
For the prediction of events in EDMD simulations an overlap function $f_{ij}(t)$ between particles i and j is defined, where the squared quantities are used merely because they are easily accessible.\\
\begin{align}
f_{ij}(t) & \coloneqq  | \vec{r}_j(t) - \vec{r}_i(t) |^2 - \sigma^2\\
          & \; \; \, \vrule
  \begin{aligned}[t]
    \quad \text{with} \quad \vec{r}_i(t) &= \vec{r}_i(t_0) + (t-t_0) \; \vec{v}_i(t_0) \, \text{,}\\
    \Delta t \coloneqq & \; t-t_0  \, \text{,} \\ 
    \vec{v}_{ij}(t) &\coloneqq  \vec{v}_j(t) - \vec{v}_i(t) \, \text{,}\\
    \vec{r}_{ij}(t) &\coloneqq  \vec{r}_j(t) - \vec{r}_i(t) \, \text{,}\\
    \Leftrightarrow \quad \vec{r}_{ij}(t) &= \vec{r}_{ij}(t_0) + \Delta t \; \vec{v}_{ij}(t_0)
  \end{aligned}\\
f(t)  & = ( \vec{r}_{ij}(t_0) +  \Delta t \;  \vec{v}_{ij}(t_0))^2 -\sigma^2 \\
\label{eqn:overlap_f}
f(t)  & = |\vec{r}_{ij}(t_0)|^2 + \Delta t ^2 \; |\vec{v}_{ij}(t_0)|^2 - 2 \Delta t \; \vec{r}_{ij}(t_0) \cdot \vec{v}_{ij}(t_0)  -\sigma^2
\end{align}  

The overlap function has the property that it is negative for two particles being closer than their diameter, 0 for at collision and positive if neither overlapping nor touching. The calculation of the next collison thus is to calculate the roots of \autoref{eqn:overlap_f}.\\

%Solving it \todo{go to the library and take again the book to write from it the solution}
Solving for $\Delta t$ with $|\vec{r}_{ij}(t_0)|^2 \coloneqq rr $, $|\vec{v}_{ij}(t_0)|^2 \coloneqq vv $ and  $\vec{r}_{ij}(t_0) \cdot \vec{v}_{ij}(t_0) \coloneqq rv $ is rather trivial:
\begin{align}
0 &= rr + vv \; \Delta t ^2  - 2 rv \; \Delta t  -\sigma^2\\
\Leftrightarrow \quad 0 &= \Delta t ^2 - \frac{2rv}{vv} \; \Delta t + \frac{rr - \sigma^2 }{vv}\\
\Leftrightarrow \, \, \Delta t &= - \frac{rv}{vv} \pm \sqrt{\left(\frac{rv}{vv}\right)^2 - \frac{rr - \sigma^2 }{vv}}
%\label{eqn:collision_prediction_pre}
%\Leftrightarrow \, \, \Delta t &= \frac{ - rv \pm \sqrt{ (rv)^2  - vv (rr - \sigma^2 )} }{vv}\\
%\Leftrightarrow \, \, \Delta t &= \frac{rv \pm \sqrt{ (rv)^2  - vv (rr - \sigma^2 )} }{vv} \; \frac{ - rv \mp \sqrt{ (rv)^2  - vv (rr - \sigma^2 )}}{ - rv \mp \sqrt{ (rv)^2  - vv (rr - \sigma^2 )}}\\
%\label{eqn:collision_prediction}
%\Leftrightarrow \, \, \Delta t &= \frac{(rr - \sigma^2 )}{ - rv \mp \sqrt{ (rv)^2  - vv (rr - \sigma^2 )}}
\end{align}
%To circumvent a caveat when executing on a floating point machine, \autoref{eqn:collision_prediction_pre} is rewritten as in \autoref{eqn:collision_prediction}.\\ 

But a caveat when executing on a floating point machine is present as can be seen when considering which solution is of interest. As for a possible collision it is necessary that the two particles move towards each other we can conclude that the scalar product is required to be negative $rv<0$, because otherwise the particles are already moving away from each other.\\ 

Also the quadratic formula has two solutions, corresponding to the entry and the exit of the overlap. Because the entry has to be prior to the exit, we further conclude that interest lies on the smaller solution that is:
\begin{align}
\label{eqn:collision_prediction_pre}
\Delta t &= \frac{ - rv - \sqrt{ (rv)^2  - vv (rr - \sigma^2 )} }{vv}
\end{align}
Now for the case where the distance of the spheres is already close to the diameter of the spheres we find $(rv)^2 \gg (rr-\sigma^2)$, which results in a cancelation of two large numbers leaving a small number. Floating point number operations are inherently bad suited because they tend to large inaccuracy in this case. Rewriting \autoref{eqn:collision_prediction_pre} by making use of the third binomial formula \todo{look if this is fine to write.} leads to:
\begin{align}
\label{eqn:collision_prediction}
\Delta t &= \frac{(rr - \sigma^2 )}{ - rv + \sqrt{ (rv)^2  - vv (rr - \sigma^2 )}}
\end{align}
Comparably \autoref{eqn:collision_prediction} does not contain a cancelation of the type seen before and such is better suited for the use in a computer simulation. \todo{cite Goldberg '91}\\

%The algorithm proposed by Bannermann\cite{Bannerman2014} works by differentiating 4 cases:
%\begin{description}
%\item[First,] \hfill \\ if $rv>0$ the particles move away from each other leading to a collision time of $\Delta t = \infty$.
%\item[Second,]\hfill \\ if $rr<\sigma^2$ an overlapis present resulting in an immediate collision time of $\Delta t = 0$
%\item[Third,] \hfill \\ if $(rv)^2  - vv (rr - \sigma^2 ) \leq 0 $ the two particles miss each other, including touching without momentum transfer, resulting in a collision time of $\Delta t = \infty$
%\item[Fourth,] \hfill \\ if none of the before is given the particles collide and $\Delta t$ is calculated by \autoref{eqn:collision_prediction}.
%\end{description}

The event prediction algorithm proposed by Bannermann\cite{Bannerman2014} works by differentiating 4 cases:
\begin{enumerate}
\item If $rv>0$ the particles move away from each other leading to a collision time of $\Delta t = \infty$.
\item If $rr<\sigma^2$ an overlap is present resulting in an immediate collision time of $\Delta t = 0$
\item If $(rv)^2  - vv (rr - \sigma^2 ) \leq 0 $ the two particles miss each other, including touching without momentum transfer, resulting in a collision time of $\Delta t = \infty$
\item If none of the before is given the particles collide and $\Delta t$ is calculated by \autoref{eqn:collision_prediction}.
\end{enumerate}

All collision times for a particle are then stored in a queue sorted by event time called particle event list (PEL). From the PEL the first entry is then passed to the global FEL.\\
This procedure initally takes place for all particles to set up the system and later on for particles involved in an event after its execution.\\ 

As will be discussed in \autoref{sec:implemetation} some widely used measures like reducing redundant calculations or implementing a cell system to reach $\mathcal{O}(N)$ computation timehave been implemented.\\

A further detail to take care of is the possibility of scheduled events which have bcome invalid due to a earlier collision of the one of the particles. This is handeled by assigning an interaction count to each particle and then store this at precalculation time with the event. When the event comes up, and the interaction count of one of the particles has increased in the meantime, the event is said to be invalidated. Depending on which particle had an event in the meantime the invalidation either causes no action or a recalculation of new events.\\  

\subsection{Details of the Implementation} 
\label{sec:implemetation}
%Add Details of for example FEL, and backupevent handling, double time precision, reset sim
As the simulation code is based on an earlier Monte Carlo Code for hard spheres a complete walk through the whole progam would become quite extensive. Such we will focus on key points to understand the details of the simulation.\\

\subsubsection{\textit{Event} struct}
\label{sec:event_struct}
We start with the basic \textit{Event} struct which includes 6 entries as shown in \autoref{tab:event_entries}.
\begin{table}[h!]
\centering
\begin{tabular}{c|c}
\textbf{Datatype} & \textbf{Name of entry}\\ \hline
(timeType) & time \\
(int) & event\_type \\
(particle*)  & particle \\
(void*) & partner \\
(int) & particle\_count \\
(int) & partner\_count \\
\end{tabular}
\caption{Content of the \textit{Event} struct.}
\label{tab:event_entries}
\end{table}
The type of \textit{time} (timeType) is usually set to double. The \textit{time} variable itself represents the time for when the event is scheduled.\\ 
The \textit{event\_type} variable is either set to 0 or 1 and indicates if the event is a cell transfer or a collision of two particles.\\
The \textit{particle} variable is a pointer to the particle for which the event has been pre calculated, while the \textit{partner} variable is set to be a void pointer. Such it is possible to either interpret it as a particle pointer for the collision type event or as an interger pointer to the index in the current cells' neighbours list for transfer events.\\
In the last two rows the interaction counts for particle and partner are listed as well. As the destination cell in a transfer event does not require an interaction count, the \textit{partner\_count} variable is only used for collision events.\\

The \textit{event} struct is used for all events throughout the simulation. For read and write operations with the HDF5 file format, the struct \textit{event\_data} is available which uses only indexes instead of pointers.\\

\subsubsection{\textit{Particle} class}
\label{sec:particle_class}
The \textit{Particle} class is comparably to the one from the MC code basis. Its MC related variables have been removed and additional key variables and concepts will be discussed in the following:\\

First a vector storing events called \textit{backupEvent} has been added to make it possible to store events from the pre calculation for the case of the first event being invalidated. The idea of reusing events is discussed in many publications, for example that the memory cost increases onyl moderately with more backup events while the speedup does not increase much for more than two stored events \cite{Bannerman2011}. It also has been argued that the added complexity can not account for the increase in efficency\cite{DONEV2005}.\\ 
Eventhough in the own simulations a decrease in calculation time of more than 10\% was observed and the cost of complexity was seen as moderate. The difference might be explained by the fact that the systems under consideration in this thesis have a rather large particle density, leading to more invalid collsions.\\

In the context of reusing pre calculated results, it should also be mentioned that after a cell transfer the recalculation of events can be restricted to partner particles only in the new neighbouring cells, leading to only 1/3 of the calculation time in this case. But as mentioned systems under consideration are mostly rather dense and such the number of transfers is often at below 5\%. Thus the increase in efficency was assumed to be to costly on the complexity side, and not implemented. Eventhough for sparse systems, it might make sense to include an \textit{updatePEl} routine.\\


Also key differences to the former MC Particle type are the variables \textit{total\_interactions} and \textit{particle\_delayed\_time}. The first is the variable for book keeping of interactions, while the second represents the event driven character of the simulation. Because each particle only moves on purely ballistic trajectories until an event occurs, it is not necessary to keep all particle positions and velocities synchronized in time. Quite on the contrary it would mean executing extra operations together with summing rounding errors by each floating point operation.\\

Because sometimes it is desired to have the whole configuration at one point of time, the \textit{transferToTime()} function of the particle provides the possibility to take the particle into the present. This is necessary soon as measurements are performed on the system, including snapshots.\\

As mentioned before the system behaves chaotic even under slightest changes like a rounding error from an extra floating point operation. A result of this is that measuring at different rates during a simulation changes the simulation trajectory quite a bit. It has been observed that such a system may keep close to the undisturbed trajectory for about 50-100 events/particle. As it is of desire to measure quantities and take snapshots without disturbing the simulation, the simulation program makes employs copies of the configuration being costly in terms of memory but making simulation resets or higher sampling rates at interesting points possible within a defined trajectory.\\ 

The structure is as follows: The first copy is rewritten with an image of the working configuration just before any measurement. The working trajectory iteself is then disturbed by the measurement, and afterwards replaced with its state before the measurement from the backup configuration.\\
The second copy actually includes the full simulation state, while the first only includes the particle cofiguration. This second one might be used to save a state during the simulation and reset to just the same point at any later time.\\


\subsubsection{The \textit{Box} class}
\label{sec:box_class}
The box of the simulation stayed mostly the same as in the previous MC code. One chanege is the array \textit{neighbours\_lookup} which has been added. It contains the indices for the cells' \textit{neighbours} array pointing to cells that share their surface. It is used to identify which cell a particle has to be transfered to during a cell transfer event.\\

Furthermore the \textit{Update} routine now takes care of all quantities depening on the length of the box, making the \textit{rescale} routine a simple rescaling of the edge lengths with an additional \textit{Update} command.\\

\subsubsection{The \textit{Scheduler} class}
\label{sec:scheduler_class}
While the afore mentioned parts of the program are necessary for the EDMD program, the \textit{Scheduler} class contains the most distinct parts of the program. It keeps track of all events to come, predicts new events and orchestrates the execution of the events. The essential functions are discussed in the following subsections while some basic properties are shortly highlighted here.\\

First of all the \textit{Scheduler} holds the Future event list (FEL) in which at least one event per particle is stored. As discussed within \autoref{sec:particle_class} the simulation is capable of saving the complete state of a trajectory, including all pre calculated events. For this purpose an array of \textit{Events} is available.\\
Furthermore the \textit{Scheduler} includes the \textit{gloabl\_time}.\\
Important for the efficency is the pre allocation of all arrays used within the prediction calculations, as the number of executions for the collision prediction routine is about $\frac{30}{\text{particle} \cdot \text{step}}$ accounting to a few billion function calls during a small simulation.\\

\subsubsection{\quad \textit{Scheduler::predictTransfer()}}
As the name suggests this function predicts the next cell transfer of a particle due to its movement. For this it calculates the position of the particle at global time, which for a valid state of the simulation always lies within its cell. By transforming the momentary position of the particle from the global coordinate system to the coordinate system of the cell and taking into account the periodic boundary conditions, we can write for each dimension $i$ the equations
\begin{align}
t_{i1}=-\frac{r_i}{v_i} \quad \text{and} \quad t_{i2}=\frac{l_i-r_i}{v_i}
\end{align}
which describe the times when the particle pierces the cell's left and right boundaries. A negative time corresponds in this case to a boundary crossing in the past, a time comparable to 0 means that the particle is on the edge of its cell and a positive time means that the boundary crossing lies in the future. By going through the different possible cases for $t_1$ and $t_2$ we find the resulting next crossing time for each case as shown in \autoref{tab:crossing_times}.
\begin{table}[h]
\centering
\begin{tabular}{c|c||c|c}
$t_1$ & $t_2$ & Result & Case \\ \hline
> & > & invalid & - \\ \hline
> & = & $t_{\text{crossing}} = t_1$ & 0 \\ \hline
> & < & $t_{\text{crossing}} = t_1$ & 1 \\ \hline
= & > & $t_{\text{crossing}} = t_2$ & 2\\ \hline
= & = & invalid & - \\ \hline
= & < & $t_{\text{crossing}} = t_1$ & 3 \\ \hline
< & > & $t_{\text{crossing}} = t_2$ & 4 \\ \hline
< & = & $t_{\text{crossing}} = t_2$ & 5\\ \hline
< & < & invalid & - \\ \hline
\end{tabular}
\caption{Possible results for left and right crossing time with resulting choice of next crossing time. >, = and < are to be read as for example $t_1 > 0$. The case indicates the case number within the actual simulation.}
\label{tab:crossing_times}
\end{table}

By collecting the next crossing times for each dimension and taking the minimum of these times the exit time of the particle from its cell is determined.\\

The return value of the routine is an \textit{Event} where the partner is given as an address to the box' \textit{neighbours\_lookup}. The index lies between 0 and 5, corresponding to the 6 possible neigbhour cells sharing a surface with the current cell of the particle. Each valid case represents a distinct neigbour cell and the index within the cells \textit{neighbours} array is clearly defined by the cell setup routines and is shown in \autoref{tab:cell_neighbour_index}. 

\begin{table}[h]
\centering
\begin{tabular}{c|c|c|c}
dimension & boundary & case & index \\ \hline
x & front & 0 & 12 \\
 & back & 1 & 13 \\ \hline
y & front & 2 & 10 \\
 & back & 3 & 15 \\ \hline
z & front & 4 & 4 \\
 & back & 5 & 21 \\
\end{tabular}
\caption{Overview of the cells' \textit{neighbours} indices directly sharing a surface for 3 dimensions. As the indices hardly follow any simple pattern they are explicitly noted at this point. Obviously the cell consists of a front and a back boundary in each dimension. The corresponding case matches the one from \autoref{tab:crossing_times}.}
\label{tab:cell_neighbour_index}
\end{table}
\FloatBarrier

\subsubsection{\quad \textit{Scheduler::predictCollision()}}
The prediction of collision times has already been discussed in \autoref{sec:EDMD}. The implementation in the program first calculates all necessary scalar products while accounting for the periodic boundary conditions, and in a second step returns the collision time depending on the case at hand.\\

The presented algorithm is only valid for single sized particles. If polydisperse systems are supposed to be considered the algorithm has to be adjusted. \todo{mayhap do it in the appendix?}\\

As this routine is executed through out the simulation very often it has been tried to optimize its efficency multiple times. For example calculating only necessary results for the next case differentation has been implemented but no significant increase in efficency was recognized and for better readability the prior version has been reestablished. In either case if an optimized way of calculating the results is found it might be useful to use them.\todo{<-- Not really nice}\\    

\subsubsection{\quad \textit{Scheduler::setupFEL()}}
This routine fill the FEL of the simulation. For this purpose it iterates through all particles and calls \textit{setupPEL} for each of them. The PEL in turn is set up by predicting the next cell transfer as well as the next collisions with all particles within the $3^d$ cells directly surrounding the particle. From all predicted events only such with finite times are then written to the \textit{backupEvents} vector corresponding to the PEL of the particle.\\
For the FEL only the top event of each particle is then used. Because other events from the PEL are able to move on to the FEL ounce written to the FEL an event has to be erased from the PEL.\\

\subsubsection{\quad \textit{Scheduler::executeTransfer()}}
The execution of a transfer event is accomplishd by the event particles \textit{MoveBetweenCells()} routine. The departure cell is taken as the event particles own cell. While the event partner holds the information which of the cell neighbours is the destination cell.\\

\subsubsection{\quad \textit{Scheduler::executeCollision()}}
The outcome of a collision between particle 1 and 2 with corresponding position and velocity can be derived by momentum and energy conervation \todo{cite some textbook, or better jsut write down the calculation} too be 
\begin{align}
\label{eqn:collision_result}
\vec{v}_1^{\,'} = \vec{v}_1 + \left( \frac{\vec{r} \cdot \vec{v} }{\vec{r} \cdot \vec{r}} \right) \vec{r}_{12}
\end{align}
While \autoref{eqn:collision_result} is not directly depending on the radius, an indirect dependence  by the collision time and configuration at which \autoref{eqn:collision_result} is evaluated persists.
 
\subsubsection{\quad \textit{Scheduler::executeEvent()}}
The execution of an event works in multiple steps. At first the topmost \textit{Event} is copied from the FEL where it is deleted. Next the validity of the interaction counts of both particle (\textit{cond1}) and its partner (\textit{cond2}) are evaluated. The validation is nothing else than a comparison of the interaction counts when the event was scheduled with the present interaction counts. As the conditions are used in the following flow statements they are stored as boolean type. Furthermore in the case of a transfer event the validation of the partner is not necessary but only for better readability performed.\\
It follows a distinction between 5 cases which are given by:
\begin{description}
\item[Valid transfer (\textit{event\_type}==0 and \textit{cond1}) ] \hfill \\ The transfer is executed, the global time is evolved to the event time and the particles PEL is rebuilt and its next event pushed to the FEL.
\item[Valid collision (\textit{event\_type}==1 and \textit{cond1} and \textit{cond2}) ]\hfill \\ The collision is executed, the global time is evolved to the event time and for both particpating particles new PEL's are built and each top event is pushed to the FEL.
\item[Invalid transfer (\textit{event\_type}==0 and not \textit{cond1})] \hfill \\ The particle must have had an interaction previously where a new event for it was scheduled, thus no action is taken.
\item[Invalid collision due to particle (\textit{event\_type}==1 and not \textit{cond1})] \hfill \\  The particle must have had an interaction previously where a new event for it was scheduled, thus no action is taken as for the above.
\item[Invalid collision due to partner (\textit{event\_type}==1 and not \textit{cond2})] \hfill \\  Only the partner had an interaction previously where a new event for it was scheduled, thus a new event for the particle is required. As the particle had no further interactions, the \textit{backupEvents} are still valid and its first entry is pushed into the FEL. In case no backupEvents are stored the PEL is rebuilt and its first entry pushed to the FEL instead.
\end{description}
The order of the cases might be exchanged, except for the last two. This is because the last one assumes \textit{cond1} to be true, which is guaranteed by the case before.\\

Furthermore the routine counts the number of each case, to monitor numbers of collisions, transfers and invalidated cases by type. This is not required by the simulation but can be helpful for understanding the system and simulation.\\




\section{Measurement quantities}
\label{sec:cluster_quantities}
list quantities and their use etc.

\subsection{Mean squared displacement}
\label{sec:MSD}
-Averaged or as distribution\\
-Reuqired for diffusion\\

\subsection{Radial distribution fucntion}
\label{sec:RDF}
-Well show an RDF?

\subsection{Largest Cluster}
\label{sec:lc}
- Non local\\
- large in case of nucleation\\
- requires almost no memory\\

\subsection{Cluster size distribution}
\label{sec:pnt}
- Non local\\
- requires more memory still not as much as for example particle positions\\
- 

\subsection{Cluster positions and numbers}
\label{sec:cluster_position}
- not done\\
- local as single clusters can be named and followed from formation to melting or final growth\\
- Much more information as locallity can be archieved.\\


\subsection{Tensor of Gyration}
\label{sec:ToG}
-Only done for largest cluster of size larger than some threshold\\
-How to extract the eigenvalues l1,l2,l3\\
-Radius of Gyration\\
-Anisotropy\\
-Asphericity\\








\section{Probe of simulation code}
\label{sec:probe}
To probe the simulation dynamics we measuremed the longtime diffusion constant and the radial distribution function of the stable hard sphere liquid, as there are many measurements available in the literature to compare with.

\subsection{Diffusive behaviour}
\label{sec:diffusion_probe}
The diffusive behaviour of particles in a liquid usually can be three sepearted in thee distinct parts. First the short time diffusion which can be understood as the random movement of the particles within their momentary cage within the fluid. Second a sub diffusive phase in which the particles are repelled for the first time by their nearest neighbours. And third the long time diffusion to describe the random propagation of the particle through the fluid over time.\\
   
As the ballistic hard sphere system enters into the long time diffusion almost at ounce only this is really measureable. By assuming a diffusive process we have the expectaion that the average mean squared displacement (MSD) of a particle can be well described by 
\begin{align}
\label{eqn:diffusion}
\langle x^2 \rangle(t) = 2 \, d \, D \, t  \, \text{,}
\end{align}
where $\langle x^2 \rangle$ is the expectation value of the MSD, $d$ the number of spatial dimensions, $D$ the characteristic diffusion constant, and $t$ the time by which the system has evolved.\\

By measuring $\langle x^2 \rangle (t)$ and making a linear regression to the data points we can find the Diffusion constant $D$.

To probe the simulation code, systems as characterized in \autoref{tab:system_diffusion} have been used. The equilibration phase has been carried out up to $\eta_f = 50\%$ at the final volume fraction while above $\eta_f =  50\% $ an inital volume fraction of $\eta_i = 45\%$ has been used to obtain a fluid rather than a solid after the equilibration phase. As the measurement stretches into the metastable regime, it has also been necessary to check for nucleation events which were not present during the measurements.\\

The resulting diffusion constants depending on the fluid volume fraction are shown in \autoref{fig:diffusion_const} alongside values obtained from the literature, for a similar system.\\

\begin{figure}[h]
\centering
\includegraphics[width=0.7 \linewidth]{diffusion_probe.pdf}
\caption{Logarithmic long time diffusion constant of the hard sphere liquid as measured in the own simulations as well as measurements from the literature. \todo{cite the people.}}
\label{fig:diffusion_const}
\end{figure}

As it can be seen the EDMD simulation is very well capable of reproducing the diffusion constant for the hard sphere liquid, and such we expect the dynamics of it to accurately represent the purely ballistic hard sphere system.\\


\begin{table}[h]
\centering
\begin{tabular}{c|c}
Parameter & Value \\ \hline
N & 16384 \\
eq\_steps/particle & 5000 \\
pr\_steps/particle & 20000 \\
$\eta_i$ & 5\% ... 50 \% \\
$\eta_f$ & 5\% ... 54 \% \\
\end{tabular}
\caption{Input parameters of diffusion test systems.}
\label{tab:system_diffusion}
\end{table}





\subsection{Radial distribution function}
\label{sec:RDF_prob}
A further well known quantity for the hard sphere system is the radial distribution function. As a theoretical prediction the Percus-Yevick approximation can be used to compare with, also it would be possible to compare with Monte Carlo simulations of the hard sphere system. In \autoref{fig:rdf_overview} an overview for a range of volume fractions is shown from the same simulations used in \autoref{sec:diffusion_probe}. Clearly visible is that no particles enter within the diameter of the spheres. Further for higher volume fractions the liquid shells become very well visible. At very high volume fraction we also find that  new peak arises below $r = 2 \sigma$.
\begin{figure}[h]
\centering
\includegraphics[width=0.7 \linewidth]{RDF.pdf}
\caption{Radial distribution functions for a range of volume fractions. The colouring corresponds to the used volume fraction.}
\label{fig:rdf_overview}
\end{figure}

To compare with Percus-Yevick approximation the radial distirbution function for two single volume fractions is shown with the corresponding theoretical solution in \autoref{fig:rdf_py}.
\begin{figure}[h]
\centering
\includegraphics[width=0.7 \linewidth]{RDF_percus_yevick.pdf}
\caption{Radial distribution function for the hard sphere system at a low and at a high volume fraction of the liquid together with the theoretical prediction from the Percus-Yevick approximation.}
\label{fig:rdf_py}
\end{figure}

As highlighted for example in \cite{Hansen2006} the theoretical approximation has some flaws as can be seen with $g(r)|_{r=1\sigma}$ being too low for the Percus-Yevick approximation. Eventhough we see that overall the two radial distribution functions follow each other rather closely.\\
Such we are confident that the elaborated simulation code is capable of producing accurate data in other contexts as well.\\

\todo{It seems in the hansen and mcdonal that the peak below 2 sigma is also not present in their MC g(r), is there a reason for this?}

\section{Estimate of required resources}
\label{sec:resources}
To choose system parameters senseful calculation times and file sizes of the simulation were characterized. This was of interest as the program was supposed to run on the NEMO high performance computing cluster which puts hard boundaries on calculation times which when treepassed can cause tremendous loss of data if not corretly caught by the program.\\
 
\subsection{Calculation time estimates}
\label{sec:calc_times}
The calculation time of the program was tested for a large range of different system sizes up to almost 9 million particles in the fluid state. As can be seen in \autoref{fig:calc_time} the calculation time increases proportional to the system size for the execution of a step as well as for a measurement of the fluid system. The calculation cost being of $\mathcal{O}(N)$ enables the study of large systems. Furthermore from the slope an expectation for the execution time of a single event can be deduced, as well as an expectation for the time necessary for a measurement. As discussed on the example of \autoref{fig:calc_q6q6} the dependence of the measurement routines on the largest cluster size were not seen here, as possile clusters remained rather small during these simulation times.\\

\begin{figure}[h!]
\centering
\includegraphics[width=0.7 \linewidth]{Calculation_times_measurement.pdf}
\caption{Overview of CPU time required for calculating a simulation step, consisting of an event for each particle, and a measurement of relevant quantities for the system. As assumed for a simulation algorithm with $\mathcal{O}(N)$ calculation effort, the data points can be described by a line rather well. As the CPU time is clearly related to the further workload of the CPU during the calculation it is also expected to find fluctuations if the other workload of the machine is not strictly controlled.}
\label{fig:calc_time}
\end{figure}


%Execution times for single steps have also been resolved with their distributions at 

%\begin{figure}[h!]
%\centering
%\includegraphics[width=0.7 \linewidth]{Simulation_runtimes.png}
%\caption{Histogram of CPU time required per execution of one event at different volume fractions. A small variation in the execution time is visible but the dependence is only small over a large range of volume fractions. Somewhat surprising are the outliers at about half the median calculation times for which no direct explanation could be found, except that mayhap the calculation node they were executed on did not have any further workload present at the time of execution.}
%\end{figure}
The effect of larger clusters was only investigated after problems with the runtime of the programs were traced back to these. The q6q6-order parameter routine was tested for larger clusters in a nucleating simulation with about 1 million particles within the box. As can be seen in \autoref{fig:calc_q6q6} the calculation cost of the cluster finding routine can be described with a quadratic dependence on the largest cluster. For an impression what this means we can use the calculation costs of a simulation step from \autoref{fig:calc_time} being about $t_{step} \approx \SI{10}{\mu s \per \text{particle} }$. Such the execution of one step takes about $\SI{10}{s}$ for 1 million particles. If a measurement is performed every 10th step, the calculation cost of the measurements without a large cluster remain below 10\%. But as the largest cluster grows to a few hundred thousand particles in size, the measurements can make up 30 \% and more of the calculation cost, or for a fixed number of steps, increase the calculation time by about 50 \%. This previously unseen effect lead to actual data loss as the combination of NEMO cluster policy and EDMD simulation program did not result in a save shutdown of the program after breaching the walltime limit of the NEMO cluster.\\

\begin{figure}[h!]
\centering
\includegraphics[width=0.7 \linewidth]{q6q6_calculation_time.pdf}
\caption{Calculation time of the q6q6 order parameter at an increasing largest cluster size during one nucleation, together with the quadratic best fit indicating that the q6q6 routine calculation effort can be approximated by $\mathcal{O}({N_{lc}}^2)$ where $N_{lc}$ is the size of the largest cluster.}
\label{fig:calc_q6q6}
\end{figure}


\subsection{File sizes estimates}
\label{sec:file_size}
A further important constraint for the simulations are the produced amount of data. To get an impression of the filesizes, the required memory for snapshots, reset steps and other measurements were measured prior to the actual simulations. The results for a single snapshot containing all positions and velocities of all particles as well as the size of a single simulation reset step containing all positions, velocities, the FEL, all PEL's and all delayed times is shown in \autoref{fig:file_size}. It can be seen that the file size is directly proportional to the system size which clearly expected as each particle adds a further set of positions, velocities etc. to the saved data.\\
The memory costs of other measurements have been left out of \autoref{fig:file_size} as these only amount to substantial filesizes if measurements at about each step for long simulations are done.\\  

\begin{figure}[h!]
\centering
\includegraphics[width=0.7 \linewidth]{File_size.pdf}
\caption{Overview of filesizes when a single setup on the one hand and a single full simulation on the other hand is saved for comparison reasons together with their corresponding linear regression. while the linear regression for 3 points is statistically not exceedingly senseful it still remains a useful tool to extract the slope which corresponds to the required memory per particle and snapshot or reset simulation.}
\label{fig:file_size}
\end{figure}




\section{First produced data}
\label{sec:data}
The motivation for the simulation code is based on the interest in nucleation rates of the hard sphere system at varying volume fractions. To observe a nucleation the volume fraction of hard spheres has to be changed rapidly from lower ones where the system is in the stable fluid phase to higher ones where a meta stable fluid-solid phase exists. If this metastable phase is evolved in time nucleations can be observed as stochastic distributed events. To measure those without effects originating from the handling of the simulation, some parameters were tested within reasonable ranges prior to the data production.\\
For this simulation the equilibration steps as well as the inital density before the volume quench seemed like they could introduce unwanted artefacts, and thus we performed some smaller data series to evaluate if and when these effects might come into play.\\

The used test system is characterized by the figures in \autoref{tab:system_start_parameter_test}.

\begin{table}
\centering
\begin{tabular}{c|c}
Parameter & Value \\ \hline
N & 16384 \\
eq\_steps/particle & 100 ... 20000 \\
$\eta_i$ & 5\% ... 49 \% \\
$\eta_f$ & 54 \% \\
\end{tabular}
\caption{Input parameters of test systems probing the dependence on equilibration steps and inital density.}
\label{tab:system_start_parameter_test}
\end{table}

The general behaviour of the systems is analysed by inspecting the cluster distribution over time. The mean cluster distribution is shown in \autoref{fig:pnt_mean} together with the same data smoothed by a gaussian filter matrix. The smoothing is used because in a next step the difference between the mean cluster distribution and the cluster distributions with varying simulation parameters is compared, and without smoothing at low count rates only fluctuations are visible.\\


\begin{figure}[h!]
\centering
\includegraphics[width=0.7 \linewidth]{mean_pnt.png}
\caption{Heat map of the mean cluster distribution over time. The diagram encompasses 800 trajectories of 16384 particles each. The colouring indicates the logarithm of the mean cluster occurance corresponding to a probability in the stationary case.}
\label{fig:pnt_mean}
\end{figure}

From \autoref{fig:pnt_mean} we can see how the system behaves after a volume quench into the metastable region. In the liquid rarely any clusters are present and thus directly after the quench no clusters are present either as the spatial configuration requires time to rearange into clusters. In the later evolution we see how clusters form, and soon after begin to nucleate leaving the range of the diagram.\\

To compare simulations with varying parameters the quantity defined in \autoref{eqn:pnt_delta} is used, where complications with zero values are circumvented by fixing these values below the regular signal.\\ Three samples of this comparison are shown in \autoref{fig:pnt_eq_step_comparison} and \autoref{fig:pnt_eq_step_comparison}. The colouring indicates $\Delta_{p(N,t)}$ defined in \autoref{eqn:pnt_delta}. As mentioned above the quantities $p_i(N,t)$ and $\langle p(N,t) \rangle$ have been smoothed by a gaussian filter, because the number of samples included, with 100 trajectories per series, were not sufficent to produce smooth distributions at the given sampling rate. Such without smoothing only fluctuations would be visible.\\ 

\begin{align}
\label{eqn:pnt_delta}
\Delta_{p(N,t)} = log ( | \frac{p_i(N,t)}{\langle p(N,t) \rangle} -1 | )
\end{align}

\begin{figure}[h!]
\centering
\includegraphics[width=0.7 \linewidth]{pnt_comparison_eq_steps.png}
\caption{Heat map of differences between the cluster distributions within simulations carried out with varying the length of the equilibration phase. The quantity used for colouring is defined in \autoref{eqn:pnt_delta}, where yellow indicates a large difference while blue indicates a small difference. Providing a legend of the colouring is ommited as $\Delta_{p(N,t)}$ has no further use as to indicate differences and actual values do not add any use.}
\label{fig:pnt_eq_step_comparison}
\end{figure}


\begin{figure}[h!]
\centering
\includegraphics[width=0.7 \linewidth]{pnt_comparison_rho.png}
\caption{Heat map of differences between the cluster distributions within simulations carried out with varying the volume fraction of the liquid during the equilibration phase. The quantity used for colouring is defined in \autoref{eqn:pnt_delta}, where yellow indicates a large difference while blue indicates a small difference. Providing a legend of the colouring is ommited as $\Delta_{p(N,t)}$ has no further use as to indicate differences and actual values do not add any use.}
\label{fig:pnt_rho_comparison}
\end{figure}

On first sight none of them differ in their general behaviour. Because at t=0 after the quench no clusters have formed yet and also no clusters were present in the stable liquid, the difference between all simulations is zero, indicated by the blue region in the top left corner. The features visible on the edge between the zero region and the nonzero region on the other side are the same, because they are features of the mean distribution carried through. Actual differences not due to fluctuations can only be seend within the green and yellow non-zero region, but none such differences is observed.\\

While it seems like the inital volume fraction of $\eta=0.4$ and $eq\_steps = 5000$ include less irregular fluctuations, dramatic effects from choosing the simulation parameters can be excluded. Intresting in this context are esspecially the simulations with $eq\_steps = 100$ because after executing 100 events/particle on average, the inital perfect crystal configuration is only on the verge of not being detected anymore. Such one could expect that these configurations might be stoill close to crystalization, but instead we do not detect any significant difference.\\

A further more quantifying analysis of the differences is given by calculating the mean nucleation rates assuming classical nucleation theory. This is done for the data shown in \autoref{fig:comparison_nucleation_rates}. The calculations of the rates have been carried out as described in \autoref{sec:induction_times}.

\begin{figure}[h!]
\centering
\includegraphics[width=0.7 \linewidth]{nucleation_rate_comparison.pdf}
\caption{Comparsion of nucleation rates under CNT assumptions for different inital densities during equilibration with eq\_steps fixed at 5000, as well as varying eq\_steps with $\eta_i$ fixed at 0.45. }
\label{fig:comparison_nucleation_rates}
\end{figure}

As we see no significant difference in the nucleation rates can be observed even for the bold setting of only 100 events per particle for the equilibration phase. Eventhough it can be seen that for this special case the rate is a little higher, but it may as well be a stochatic flucutaion.\\

Overall we conclude in this chapter that as long as parameters are set within reasonable boundaries, we expect not to have systematic influences of simulation parameters. 


%\Floatbarrier

\section{Possible extensions}
\label{sec:simulation_ext}
The program at this state is capable of simulating large systems including compression and relaxation. Extensions can aim at either increasing complexity or size of the systems. 

\subsection{Varying radius}
\label{sec:extension_radius}
A further notch of complexity would be to include polydispersity into the simulation. For this purpose the prediction of collisions has to be adjusted. When looking at the derivation of \autoref{eqn:collision_prediction} it is found that $\sigma$ being the former diameter of a sphere in the monodisperse case has to be changed to $\sigma=R_i+R_j$. In the equations the same defiitions of scalar products are used as before in \autoref{sec:EDMD}.

\begin{align}
\label{eqn:collision_prediction}
\Delta t &= \frac{(rr - \sigma^2 )}{ - rv + \sqrt{ (rv)^2  - vv (rr - \sigma^2 )}}
\end{align} 

For a physical model in which the particles are made of some matter with constant density the change of the radius is also accompanied by a change of the mass. This has to be taken into account when assigning the velocities after a collision as written in \autoref{eqn:var_mass}.

\begin{align}
\label{eqn:var_mass}
\vec{v}_i{\,'} = \vec{v}_i + \frac{2 m_j \; (rv)}{(m_i + m_j) \sigma^2} \cdot (\vec{r}_j - \vec{r}_i) \nonumber \\
\vec{v}_j{\,'} = \vec{v}_j + \frac{2 m_i \; (rv)}{(m_i + m_j) \sigma^2} \cdot (\vec{r}_j - \vec{r}_i)
\end{align}


\subsection{Muliprocessing}
\label{sec:extension_MP}
It further would be possible to precalculate different events on different processors, thereby speeding up simulations. This requires the execution of events not in the strict time order used for the EDMD algorithm shown before. Instead a check should be implemented to only execute events which are spatially seperated by some distance. This might make it possible to change only little of the dynamics. Eventhough a problem conserning this idea is that the simulation code would be required to consume less memory as this would be the bottleneck of the system sizes.\\
Such rather large changes would be necessary to impelemnt parrallel EDMD simulations. Either way it could help making system of about 100 million particles feasible.   



\newpage

\chapter{Data Analysis}
% !TEX root = writing_version.tex

\label{chp:data_analysis}
\section{Parameter choice of the simulated system}
\label{sec:system_choice}
As an integral part of this work large scale simulations have been executed on the NEMO High performance computation (HPC) cluster. The input parameters of the simulated systems are given in \autoref{tab:system_1m}.

\begin{table}[ht]
\centering
\begin{tabular}{c|c}
Parameter & Value \\ \hline
N & 1048576 \\
eq\_steps/particle & 1000 \\
pr\_steps/particle & 20000  ... 60000 \\
$\eta_i$ & 45.0 \% \\
$\eta_f$ & 53.1\% ... 53.4 \% \\
\end{tabular}
\caption[Simulation parameters of data production systems]{Input parameters of large scale simulations on the NEMO HPC cluster. The varying steps during production come by the fact, that 20000 steps were estimated to be calculated within 3 days leaving 1 day of buffer to the hard wall time limit of 4 days. Due to the increasing calculation cost of the q6q6 cluster routines for large clusters the wall time limit was still breached and without proper reset steps the datasets could not be restarted without large calculation overhead as all lost data has to be replaced, and the broken reset steps within the files would have to be removed prior to further simulations. Therefore the last proper version of the files were used resulting in varying simulation lengths but only rarely without nucleation event in the case of early breakdown.}
\label{tab:system_1m}
\end{table}

The simulations comprise four series' at volume fractions of $\eta = 0.531,\;0.532,\;0.533 \text{ and }0.534$ where each series consists of 500 trajectories. Therefore at each volume fraction a total number of about half a billion particles have been simulated in the metastable fluid.\\

The volume fractions have been chosen to probe nucleation rates to the lowest possible limit. As single nucleations have been observed down to volume fractions of $\eta=53.2\%$, the lowest volume fraction was set to just below this value as the large statistic of 500 trajectories was expected to still yield enough nucleation events to measure their rate.\\

The size of the systems was chosen comparably large with about 1 million particles. These large systems intuitively seem to be in conflict with the long induction times, but using CNT as a guideline it can be shown that the computational effort for simulating nucleation events does not increase significantly with increasing system size. As the calculation time per unit of simulation time is proportional to N, it is at a given volume fraction also proportional to the volume V:
\begin{align}
\label{eqn:system_size}
\frac{T_{CPU}}{\delta t_{Sim}} \propto N \propto V 
\end{align}

Further we expect the nucleation time in terms of the system time $\langle \tau_{Nucleation} \rangle$ to be proportional to the inverse of the system volume if assuming a nucleation rate density independent of time:
\begin{align}
\langle \tau_{Nucleation} \rangle \propto \frac{1}{V}
\end{align}

As the required CPU time for a nulcleation event is simply proportional to the product of $\langle \tau_{Nucleation} \rangle$ and the calculation time per unit of system time we can conclude:

\begin{align}
\langle T_{CPU} \rangle \propto  \frac{T_{CPU}}{\delta t_{Sim}}  \cdot \langle \tau_{Nucleation} \rangle \propto \frac{V}{V} = \text{const.}
\end{align}

Thus the size of the system is only relevant to be chosen smaller if ordering processes are important for the system, as the initial induction time would be independent of the system size. This might be the case for polydisperse systems, but in the monodisperse case the above reasoning was found to hold true.\\
An other objective that has to be considered is that less configuration snapshots of the system can be stored, as these requrie a lot of memory space. If one is interested in quantites like $g(r)$ this is not a problem as the necessary statistics can be either derived from a large set of small snapshot or from a small set of large snapshots, but for example resolving and storing the dynamcs of a configuration for a growing cluster would require using smaller system sizes as files easily grow to many GB's in size. 

\section{Long time diffusion time scale}
\label{sec:diffusion_metastable_liquid}
Diffusion or more precisely self-diffusion, characterizes the movement of the single particles within the system. While the diffusive behavior often can be subdivided into different regimes with different physical meanings as discussed in \autoref{sec:diffusion_probe}, only the long time diffusion constant is measured for the hard sphere system. To circumvent finite size effects we use unwrapped coordinates in which case the long time movement of the particles is governed by the relation \autoref{eqn:einstein_relation} which was first described by Einstein 1905\cite{Albert1905}.

\begin{align}
\label{eqn:einstein_relation}
D^S_L = \underset{t\rightarrow \infty}{\text{lim}} \frac{\langle (\vec{r}(t) - \vec{r}(0) )^2 \rangle}{2 d t}
\end{align}

With $D^S_L$ the long time self-diffusion constant which will in the following be denoted only by $D_L$, $\vec{r}(t)$ the position of a particle at time t, d the number of spatial dimensions of the system and $\langle ... \rangle$ the expectation value of the ensemble.\\

The average is measured by saving a reference position of all particles at one point, and furthermore carrying a set of unwrapped positions through the simulation. The squared distance between reference position and unwrapped position is averaged over all particles and used as the measurement of the ensemble average. Especially for large system of 1 million particles, this quantity has only very small fluctuations as can be seen in \autoref{fig:diffusion_comparison}, where the slopes of the linear regressions to the MSD trajectories are depicted.\\

\begin{figure}[ht]
\begin{center}
\subfloat[Histograms of the slopes for the linear regressions to the mean squared displacements. The histograms are for $\eta=0.531,\;0.532,\;0.533,\;0.534$.]{\includegraphics[width = 0.45 \textwidth]{diffusion_histogram_comparison.pdf}} \hspace{0.5cm}
\subfloat[Mean of the histograms with the uncertainty on the mean given by $\sigma_{\langle D_L \rangle} = \sigma_{D_L}/\sqrt{n}$ with n being the number of measurements included in the average.]{\includegraphics[width = 0.45 \textwidth]{diffusion_comparison.pdf}}  
\caption[Long time self-diffusion constant measurements from production data]{Comparison of long time self-diffusion constants at different volume fractions as histograms and means with uncertainty. As we see the }
\label{fig:diffusion_comparison}
\end{center}
\end{figure}

The diffusion coefficients are used to make the time scales of different experiments comparable. It is based on the idea that the fundamental mechanisms for nucleation and cluster growth do not vary between different hard sphere like systems, but are only scaled by the varying diffusion times. Furthermore there are theoretical predictions for the relationship of short time and long time diffusion, making it possible to compare experiments where the short time diffusion behavior is better accessible with the ballistic simulations where only the long time diffusion constant is measurable.\\


As we see in \autoref{fig:diffusion_comparison} the diffusion constants can be measured at high precision with a relative standard deviation of $\sigma_{D_L}/D_L \approx 1\%$. Hence it does not introduce large uncertainties when normalizing time related quantities by the diffusion time $\tau_{D_L} = D_L^{-1}$.

\section{Cluster size distribution over time}
\label{sec:pnt}
The cluster size distribution of the system can be used to test the assumption of Markovian dynamics by trying to find a Fokker-Planck equation describing the time evoluation of the distribution. This has been done for the Lennard-Jones system by Kuhnbold et al. 2019\cite{Kuhnbold2019}. Testing the trajectories shown in \autoref{fig:pnt_short} and \autoref{fig:pnt_long} in a similar fashion would yield a good comparison but due to time constraints of this thesis it is not done. We still can illustrate some characteristics of the metastable fluid directly after and long after the quench as it compactly shows some main features of the systems behavior.\\ 

The cluster size distributions are the averages over all trajectories at a given volume fraction. While they are normalized by the number of included measurements they have not been normalized by the volume. The maximum cluster size is set to 160 as above this value only nucleating trajectories can be seen. Also a logarithm to the base of 10 is used, and cluster sizes not present at a given time step have been fixed to a value below the minimal signal as the logarithm requires non zero values.\\
The logarithm is used because the mesurements span orders of magnitude and further it then can be interpreted as a quantity proportional to a free energy. This is justified by assuming that the cluster size distributions represents the correpsonding probability distribution and that stationary states may fluctuate in a free energy landscape where the probability for a particular state with some energy $\Delta E$ is given by a Boltzmann distribution $p\propto \exp \left( - \frac{\Delta E}{k_B T} \right)$ from which follows that $\log(p) \propto \Delta E$.\\

\begin{figure}[ht]
\begin{center}
\subfloat[$\eta = 53.1\%$]{\includegraphics[width = 0.49 \textwidth]{pnt_531_short.png}} \hspace{0.0cm}
\subfloat[$\eta = 53.2\%$]{\includegraphics[width = 0.49 \textwidth]{pnt_532_short.png}}\\
\subfloat[$\eta = 53.3\%$]{\includegraphics[width = 0.49 \textwidth]{pnt_533_short.png}} \hspace{0.0cm}
\subfloat[$\eta = 53.4\%$]{\includegraphics[width = 0.49 \textwidth]{pnt_534_short.png}}
\caption[Cluster size distributions over time after quench]{Decadic logarithm of cluster size distributions for different volume fractions in the initial phase after the quench.}
\label{fig:pnt_short}
\end{center}
\end{figure}

In \autoref{fig:pnt_short} we can see the initial phase after the quench. As the fluid before the quench was at a volume fraction of $\eta=45\%$ only very little local ordering is present directly after the quench. This changes within the first $15 -25 \delta t$ after which the distribution becomes stable, where the exact length depends on the volume fraction. This might be explained by assuming that the initial phase is how long it takes for the system to build up the local ordering in the metastable liquid and as the clusters tend to be larger for higher volume fractions more particles are required to find their ordering.\\   
To further compare the system time with the more intuitive number of collisions per particle we can use that at the given volume fraction we find  $1\delta t \approx \frac{60 events}{particle}$. When further using a collision probability of $\sim 40 \%$ for each executed event, we find that $1\delta t \approx \frac{25 collisions}{particle}$. As a result we can conclude that it takes a few hundred collisions for each particle to build up the local ordering with unstable clusters.\\

\begin{figure}[ht]
\begin{center}
\subfloat[$\eta = 53.1\%$]{\includegraphics[width = 0.49 \textwidth]{pnt_531_long.png}} \hspace{0.0cm}
\subfloat[$\eta = 53.2\%$]{\includegraphics[width = 0.49 \textwidth]{pnt_532_long.png}}\\
\subfloat[$\eta = 53.3\%$]{\includegraphics[width = 0.49 \textwidth]{pnt_533_long.png}} \hspace{0.0cm}
\subfloat[$\eta = 53.4\%$]{\includegraphics[width = 0.49 \textwidth]{pnt_534_long.png}}
\caption[Cluster size distributions for long waiting times]{Decadic logarithm of cluster size distributions for different volume fractions during the waiting time.}
\label{fig:pnt_long}
\end{center}
\end{figure}

The diagrams in \autoref{fig:pnt_long} show a zoomed out version of the same data depicted already in \autoref{fig:pnt_short}. We see that the distribution that is reached at the end of the initial phase remains stable over prolonged periods of time. Only the nulceation events which account for most of the probability at largest cluster sizes indicate that this is not a stable process but only a metastable one. Nevertheless this does not mean that it simply can be viewed as a stationary process, because when taking the ensemble as a whole, at any point of time phase transitions take place at various parts of the system.

\section{Autocovariance functions of largest cluster in the metastable fluid}
\label{sec:acf}
The autocovariance function (ACF) of the largest cluster contains information about how long a single cluster persists as the largest cluster within the volume. This is because fluctuations of clusters at different points of the volume are expected to be independent of each other and only the size of a distinct cluster should be correlated in time.\\

The autocovariance function is defined by \autoref{eqn:definition_acf} where $N_{\text{lc}}(t)$ is the number of particles in the largest cluster at time t, $\langle N_{\text{lc}} \rangle_t$ is the corresponding average over time and thus $X(t)$ describes the deviations from the average. The autocovariance function furthermore is normalized by ${ \langle X^2  \rangle }$, the variance of the data, such that $ACF(0) = 1 $.

\begin{align}
\label{eqn:definition_acf} 
ACF(\tau)=\frac{ \langle  X(\tau) \cdot  X(0) \! \: \rangle }{ \langle X^2  \rangle }\\  
\text{with } X(t)=N_{\text{lc}}(t)- \langle N_{\text{lc}} \rangle_t 
\end{align}

The ACF is calculated from the largest cluster measurement for each trajectory. Because after a nucleation event the largest cluster size surely is correlated in time, only those parts of the measurements that did not involve strong cluster growth are used. Therefore the ACF's in \autoref{fig:acf} show the temporal correlations of the largest cluster in the metastable fluid.\\

\begin{figure}[ht]
\begin{center}
\subfloat[$\eta = 0.531$]{\includegraphics[width = 0.4 \textwidth]{acf_lc_531.png}} \hspace{0.5cm}
\subfloat[$\eta = 0.532$]{\includegraphics[width = 0.4 \textwidth]{acf_lc_532.png}} \\
\subfloat[$\eta = 0.533$]{\includegraphics[width = 0.4 \textwidth]{acf_lc_533.png}} \hspace{0.5cm}
\subfloat[$\eta = 0.534$]{\includegraphics[width = 0.4 \textwidth]{acf_lc_534.png}} \\
\caption[Autocovariance functions of largest cluster in the metastable fluid]{Comparison of autocovariance functions in the metastable fluid. The top of each diagram depicts all trajectories with coloring for the largest cluster size within the used time interval. The lightest color thereby indicates a largest cluster of more than 500 hundred particles which is a nucleation event. As these are rare in the given selection, the data represents the metastable fluid well. The bottom of each diagram shows the average of the above one with decay times of $ 15 \delta t- 35 \delta t$ depending on the volume fraction. }
\label{fig:acf}
\end{center}
\end{figure}

The decay of the autocovariance functions indicates that structural fluctuations persist for longer times at higher volume fractions. From the coloring, that corresponds to the maximum cluster size within the trajectory, we can also conclude that the fluctuations tend to be larger at higher volume fractions and that for $\eta=53.4\%$ a signal from nucleation events might not be completly negligible anymore. The larger mestastable clusters were also seen before in the cluster size distributions in \autoref{sec:pnt}.\\
The time scale on which the ACF decays corresponds closely to the initial ordering time observed for the cluster distribution directly after the quench. Furthermore it also correpsonds to the lifetimes of large clusters found in the single example of the individual cluster tracking algorithm (\autoref{fig:lifetime}). This leads to the conclusion that these three observations all show the same time scale of local ordering processes within the metastable fluid. 

\section{Cluster growth and constant attachement rate}
\label{sec:cluster_growth}
Once the clusters reach a certain size they are expected to grow with new particles being attached to the surface at a constant rate leading to a growth with a proportionality of $N \propto t^3$ as shown in \autoref{eqn:constant_growth}, with k being the constant attachment rate, N the number of particles in a specific cluster, A the surface of the cluster, R the radius of the cluster and $\rho_{solid}$ the bulk density which for large clusters is a good approximation of the cluster density.\\
\begin{align}
\label{eqn:constant_growth}  
\begin{split}
\dot{N} &= A k \\
          & \; \; \, \vrule
  \begin{aligned}[t]
    \quad \text{with}  \quad  N &= \frac{4}{3} \pi R^3 \rho_{solid}\\
    \Leftrightarrow R &= \left( \frac{3 N }{4 \pi \rho_{solid} }\right)^{\frac{1}{3}} \, \text{,} \\
    \quad \text{and}   \quad A &= 4 \pi R^2 \\
    \Leftrightarrow A &= \left(\frac{4 \pi 3^2 }{\rho_{solid}^2} \right)^\frac{1}{3} N^\frac{2}{3} \, \text{,}\\
%    \quad \text{follows} \quad A &\propto N^\frac{2}{3} \\
  \end{aligned}\\
\frac{dN}{dt} &= \left(\frac{4 \pi 3^2 }{\rho_{solid}^2} \right)^\frac{1}{3}  N^\frac{2}{3} k \\
\end{split}
\hspace{1cm}
\begin{split}
%\Rightarrow \frac{dN}{dt} &= c' N^\frac{2}{3}\nonumber\\
%\vspace{0.25cm}\nonumber\\
& \!\!\!\!\!\!\!\!\! \text{From the bottom left side}\\
\Rightarrow& \quad dN \; N^{-\frac{2}{3}} = \left(\frac{4 \pi 3^2 }{\rho_{solid}^2} \right)^\frac{1}{3} k \; dt\\
          & \; \; \, \vrule
  \begin{aligned}[t]
    \quad \text{setting}  \quad  N(t=0) = 0\\
  \end{aligned}\\
\Leftrightarrow& \quad 3 N^{\frac{1}{3}} = \left(\frac{4 \pi 3^2 }{\rho_{solid}^2} \right)^\frac{1}{3} k t\\
\Leftrightarrow&  \quad N^{\frac{1}{3}} = \left( \frac{4 \pi}{3 \rho_{solid}^2} \right)^\frac{1}{3} k t
\end{split}
\end{align}  

As the systems are able to accommodate clusters up to a few hundred thousand particles and mostly just one cluster forms during a simulation, the attachment rate can be measured by a linear regression to the third root of the number of particles in the largest cluster over time. As an example this is visualized for the trajectories at $\eta=0.532$ in \autoref{fig:cluster_growth_example}.

\begin{figure}[ht]
\centering
\includegraphics[width=0.7 \linewidth]{cluster_growth_example.pdf}
\caption[Largest cluster trajectories from production data with constant attachment rates]{Trajectories of the third root of the number of particles within the largest cluster $(N_{\text{LC}})^{1/3}$ over time. Clearly visible is the linear proportionality for which a linear regression is shown together with the data. The cut of some data sets at $t \approx 300 \delta t $ is due to the trespassing of the maximum wall time of the NEMO computational cluster. Systems that hosted a nucleation event in the first simulation interval before $T \approx 300\delta t  $, contain a too large cluster in the next simulation interval leading to the breach of the wall time limit due to the quadratic effort required for the q6q6 cluster finding routine. It can be assumed that clusters forming just before $t \approx 300 \delta t$ might not have been recognized due to this flaw. But the number of trajectories concerned by this is small and the impact is not easy to recognize when looking at the induction time distributions in \autoref{fig:induction_distributions}.}
\label{fig:cluster_growth_example}
\end{figure}

Subsequently the slopes of the linear regressions have been collected in histograms shown in \autoref{fig:constant_growth_rates}. By \autoref{eqn:constant_growth} these slopes correspond to constant attachment rates with a prefractor depending on the density within the cluster, but as the densities of concern are very close to each other they only introduce a relative difference of 0.5\% between the rates of lowest and highest volume fractions. For this reason the dependence is neglected in the qualitative comparison and the constant attachement rate with its prefactor is defined as $ c \coloneqq k \left( \frac{4 \pi}{3 \rho_{solid}^2} \right)^\frac{1}{3} $. With this approximation the equation for the number of particles in a cluster over time, given in \autoref{eqn:constant_growth}, simplifies to the one given before in \autoref{eqn:simple_growth}.\\ 

\begin{figure}[ht]
\begin{center}
\subfloat[Histograms of the slopes from the linear regressions to third root of the largest cluster during the stable growth process. The histograms are for $\eta=0.531,\;0.532,\;0.533,\;0.534$.]{\includegraphics[width = 0.45 \textwidth]{const_growth_rate_histogram_comparison.pdf}} \hspace{0.5cm}
\subfloat[Mean of the histograms with the uncertainty on the mean given by $\sigma_{\langle c \rangle} = \sigma_c/\sqrt{n}$ with n being the number of measurements included in the average.]{\includegraphics[width = 0.45 \textwidth]{const_growth_rate_comparison.pdf}}  
\caption[Constant attachment rate measurements from production data]{Comparison of growth rates in the constant attachment regime.}
\label{fig:constant_growth_rates}
\end{center}
\end{figure}

What we see from the histograms is that the distribution is rather spread out, but not significantly depending on the volume fraction. Only for $\eta = 0.531$ we find a smaller growth rate. A possible explanation for this behavior could be that the growth by heterogeneous crystallization on the cluster surface leads to a higher growth rate for higher volume fractions as it is less likely for the lower volume fractions. But due to the low statistics at the lowest volume fraction it is also possible that only a statistical fluctuation is seen.\\ 
To investigate if the attachement is diffusion or reaction controlled we may note that the diffusion constants vary from $D=0.0081|_{\eta = 0.532}$ to $D=0.0075|_{\eta = 0.534}$. They span a difference of about 7.5\% but as the relative statistical uncertainty of the growth rates is of the order of 5\% it requires a larger number of samples to answer this question.

\section{Tensor of gyration evaluation}
\label{sec:tog}
The tensor of gyration is a very useful tool as it describes the second moments of the position distributions. Thus it comprises information about the spatial extent in all three dimensions with commonly defined quantities being the radius of gyration, asphericity and anisotropy, see Theodorou and Suter 1985\cite{Theodorou1985}.\\

The tensor of gyration itself is defined by
\begin{align}
\label{eqn:tensor_of_gyration}
&S_{mn}=\frac{1}{N} \sum_{i=1}^{N} r^{(i)}_m r^{(i)}_n\\
\label{eqn:center_of_mass}
&\text{with} \quad \sum_{i=1}^{N} \vec{r}^{(i)} = 0 \; \text{.}
\end{align}

As described by \autoref{eqn:center_of_mass} the matrix $S_{mn}$ is calculated in the center of mass frame for particles with the same mass. The tensor of gyration can be diagonalized, with the three Eigenvalues $\lambda_1^2$, $\lambda_2^2$ and $\lambda_3^2$ that are chosen with $\lambda_1^2 \leq \lambda_2^2 \leq \lambda_3^2 $. These three Eigenvalues correspond to the spatial extents of the cluster within the Cartesian system in which the tensor of gyration becomes diagonal. The aforementioned shape descriptors are defined in \autoref{eqn:tog_quantities1} - \ref{eqn:tog_quantities4}.

\begin{align}
\label{eqn:tog_quantities1}
\text{(squared) Radius of gyration:} \quad &R_G^2 = \sum_{i=1}^3 \lambda_i^2\\
\label{eqn:tog_quantities2}
\text{Asphericity:} \quad &b = \lambda_3^2 - \frac{1}{2}(\lambda_1^2+\lambda_2^2)\\
\label{eqn:tog_quantities3}
\text{Acylindricity:} \quad &c = \lambda_2^2 - \lambda_1^2\\
\label{eqn:tog_quantities4}
\text{Relative shape anisotropy:} \quad &\kappa^2 = \frac{b^2 + \frac{3}{4} c^2 }{R_G^4} =  \frac{3}{2} \frac{ \sum_{i=1}^3 \lambda_i^4 }{\left(\sum_{i=j}^3 \lambda_j^2 \right) ^2 } - \frac{1}{2}
\end{align}

For a better understanding of the above defined descriptors their meaning is discussed in the following.

\begin{description}
\item[Radius of gyration $R_G$] \hfill \\ An averaged radius of the structure. For a sphere with radius $R$ it is given by $R_G = \sqrt{\frac{3}{5}} R$.
\item[Asphericity b]\hfill \\ The difference of the largest extent and the average of the two smaller extents. For a sphere these are the same and the asphericity becomes zero, even though this is also the case for a cube.  
\item[Acylindricity c] \hfill \\ The difference of the two smaller extents, as for a long cylinder they are the same and the acylindricity becomes zero.
\item[Relative shape anisotropy $\kappa^2$] \hfill \\ A weighted squared sum of the asphericity and the acylindricity normalized by the fourth power of the radius of gyration to obtain a dimensionless quantity between 0 and 1. For a sphere it is zero while it becomes one in the case of all particles being aligned in a straight line.
\end{description}

To spot possible correlations between a cluster's shape and its growth, the radius of gyration, the asphericity and the relative shape anisotropy have been plotted against the cluster size and then colored by three scalar quantities characterizing the growth process of each trajectory.\\ 
The first of them is the induction time, as early nucleations might arise from less ordered clusters resulting in a higher asphericity. The second is the constant attachement rate during cluster growth, where similarly one may expect that clusters including more defects may grow slower and also be less spherical. The third quantity is an exponential initial growth rate which is used to characterize how swift the precursor grows into the later crystal, again with the intuition that clusters with a higher asphericitiy may tend to a slower initial growth as they might be less ordered. For quantifying the initial growth rate an exponential function has been fitted to the data up to a cluster size of 500 particles.\\
The representation depending on the cluster size is used to make the different trajectories comparable, as we expect similar behavior for similar cluster sizes. Because the cluster size depending on time is almost monotonic for cluster sizes above a few hundred particles, it roughly corresponds to a transformation of the time axis, while the order is only little influenced. Nevertheless it should be kept in mind that this does not constitute a function anymore.\\
Finally the number of particles, as well as the shape descriptors can span many orders of magnitude making logarithmic scales useful.\\

A large overview produced by this procedure is given in \autoref{fig:tog_overview} for the nucleated trajectories at $\eta=0.534$ with the three shape descriptors in the vertical direction and the three scalar coloring schemes in the horizontal direction.\\

\begin{figure}[!h]
\centering
\includegraphics[width=0.8 \linewidth]{Series_534_corrolation.png}
\caption[Tensor of gyration measurements from production data]{Overview of the shape descriptors radius of gyration ($R_G$), asphericity (b) and anisotropy ($\kappa^2$), depending on the size of the cluster are shown. The coloring indicates the scalar quantities induction time, constant attachment rate and initial growth rate. Further a smoothed arithmetic mean and median are included in black and green respectively.}
\label{fig:tog_overview}
\end{figure}

From the overview we get no obvious sign that there are any correlations between cluster shape and growth rates or between cluster shape and the induction time. Because of that no deeper analysis is done, but instead we conclude that by this superficial analysis we cannot relate the shape descriptors to the cluster growth. Also a similar approach for the three scalar quantities that describe the growth process has been done, but neither showing significant correlations.\\  

Nevertheless from the calculated means, especially for the anisotropy $\kappa^2$, we can see that up to a size of about 1000 particles the clusters become more spherical while at higher particle numbers this tendency towards a sphere comes to an halt. This could be explained for example by the fact that the clusters always exhibit crystal faces leading to some unavoidable asphericity. An other explanation could be that the attachement rate for one crystal face might be higher than for an other, as this also would lead to unspherical growth. But as the clusters are rather close to a sphere, the attachemnt rate would also not vary much between the different crystal faces. It also has been observed that very large single crystals of a few hundred thousand particles may only form at volume fractions of $\eta = 53.2 \%$. At higher volume fractions domains form as it seems that heterogenous nucleation takes place close to the surface of the cluster, leading to new crystal orientations that are included into the crystal.\\


\section{Nucleation time dilemma}
\label{sec:nucleation_times}
To calculate induction times or average nucleation times, we will require a definition of when a crystal is called nucleated. This means we have to define the point at which a cluster is not merely an unstable fluctuation in the liquid anymore, but instead becomes a stable crystalline solid.\\
Many definitions can and have been used used for htis purpose. For example a cluster can be defined as crystalline soon as it surpasse the CNT's critical size or a multiple of it. One can also use a committer analysis to find the size where a crytallite keeps growing with a 50:50 chance. Also often apllied is the method to rewind a trajectory with a stable crystal back to the point where the cluster's size vanishes. A further approach is to fit the growth during later times and extrapolate it to the time when the cluster vanishes.\\

All these definitions differ only by a delay $\Delta_{\tau}$ which is a distribution holding the information of how long it takes for varying clusters to pass from the first criterion to the next.\\
For example we can take as a first point the time when a cluster, known to crystallize at later times, is not distinguishable from any other structural fluctuation in the liquid i.e. when the size of the cluster is below some threshold given by the size of clusters regularly present in a given volume.\\
The second point we can set by either the critical size of CNT or by some other criterion when we are sure that the cluster has stabilized and will only continue to grow.\\
At the first of these two points, the fluctuation leading to the crystallization occurs but it would not be possible to tell yet if this precursor melts or continues to grow, while at the second point the crystal is stable. For this reason the first might be called a precursor nucleation and the second crystal nucleation. Between these two points we find the time difference to be the time it takes for the precursors to form a stable crystal. This includes also that some precursors might loiter for awhile before forming the stable phase, while others pass this gap rather directly.\\

When calculating a mean induction time, the delay $\Delta_{\tau}$ propagates also to the final result and as it is a stochastic distribution also its higher moments are propagated leading to a smaller precision. After all this means that the induction time depends directly on the definition of crystallization and they are only roughly comparable. In \autoref{fig:induction_distributions} three distributions with varying definitions for the induction time are visualized.

\begin{figure}[!h]
\centering
\includegraphics[width=0.6 \linewidth]{varying_induction_time.pdf}
\caption[Comparison of different definitions for the induction time]{Induction time distribution obtained by different definitions. While the two methods using extrapolation seem to have the two effects of smearing the signal as well as shifting them, the method of defining the nucleation as the time when the largest cluster is last below the horizon of fluctuations seems to return the most accurate and precise distribution. The final simulation time is marked by the grey line. As clusters require some time to clearly be recognized as crystals, no nucleation events are seen towards the end of the simulation interval. To counteract this we truncate the distribution in the analysis such it does not introduce a bias on the final result.}
\label{fig:induction_distributions}
\end{figure}

The three methods explicitly used here are given by the following:
\begin{description}
\item[Horizon crossing]{\hfill \\The time of nucleation is obtained by following the trajectory of the largest cluster within a nucleated system back to the point where it last crossed the average largest cluster of the metastable fluid. The name horizon crossing refers to the idea that fluctuations of the largest cluster are mostly independent, as the largest cluster is not fixed in the box, but fluctuations at different locations contribute. Only extraordinary large fluctuations will be seen for a prolonged periods of time and therefore will lead to correlated fluctuations of the largest cluster size. The crossing of the trajectory below this horizon, where it does not describe a distinct cluster anymore, is meant by the name.}

\item[Exponential extrapolation]{\hfill \\For this method an exponential growth is fitted to the largest cluster data up to N<500. Extrapolating to smaller times makes it possible to evaluate when the exponential function crossed 10 particles, which is then taken as the induction time. The method tends to find negative induction times that are not physical, but only an artifact of the method.}

\item[Constant extrapolation]{\hfill \\The name refers to the constant attachment rate found at later times for the cluster growth. It can be extrapolated to earlier times until the cluster completely vanishes. As the constant attachment rate is higher than the initial growth rate this method returns too large induction times.}
\end{description}

As can be seen the horizon crossing method returns a rather smooth distribution that also roughly can be approximated by an exponential decay that is expected for a constant nucleation rate as is shown in \autoref{sec:induction_time_expectation}.\\


\section{Induction time by exponential distribution assumption}
\label{sec:induction_times}
\todo{Some of this intorduction stuff may better fit into comparison to real world?}\\

Nucleation rates for the meatstable hard sphere fluid have been measured on the experimental as well as on the theoretical side, but with a large discrepancy as discussed in \autoref{sec:comparison}. The employed procedures and definitions also vary but not to a point to explain the discrepancy so far. The differences mostly originate from the kind of accessible system and information. While the experimentalists often have access to very large systems but without knowing all positions at all times, theorists mostly have smaller systems in numerical simulations but with the advantage of being able to access all particle positions and in case of MD simulations their velocities as well.\\
On the experimental side light scattering and optical methods are mostly employed to measure the structural properties of the probe, on the theoretical side different cluster finding algorithms are used.\\
While experimentalists may define an induction time by how long it takes for a quenched system to reach some level of overall crystallinity, theorists have often used simple approaches like the average time to nucleation for a couple of trajectories to measure their induction times for example by Filion et al. 2010\cite{Filion2010a} \todo{cite it here or not? It not such a positive statement}. This method requires the theorist to wait for all trajectories of an ensemble to show nucleation, what renders it very unsuitable for systems at low volume fractions where the induction time increases steeply.\\

To circumvent this problem we will define the nucleation rate in the following differently without requiring all simulations to nucleate. In fact we can also show that the uncertainty of the induction time obtained from the data is not significantly reduced anymore for measurements longer than the mean induction time.

\subsection{CNT expectation of the induction time distribution}
\label{sec:induction_time_expectation}
In \autoref{sec:CNT} we introduced classical nucleation theory and its constant nucleation rate depending on the barrier height in the free energy landscape. Even if there are signs that CNT is not appropriate for describing nucleation process completly, we will use its prediction of a constant nucleation rate as an assumption to define a constant scalar nucleation rate as well which can be compared to other literature values.\\

As mentioned before, in the discussion of the system sizes (\autoref{sec:system_choice}), the induction time of a system depends on the volume under consideration and for this reason it is commonly defined as a nucleation rate density $\kappa$. By using the diffusion time $\tau_L = D_L^{-1}$ as a unit of time furthermore makes the comparsion to other systems with faster or slower dynamics possible.\\

Considering a set of $m$ simulations at a given volume fraction, we can describe the total system as a sum of $m$ subvolumes, each of size $V_{box}$.  Further we can define the number of boxes in which a nucleation occurred as $n(t)$ and exclude these from the further simulation.\\

In this case the total nucleation rate $ \dot{n} $ can be written by \autoref{eqn:nuc_rate} from which in the continuous limit of an infinite number of subvolumes we can deduce the expected induction rate.

\begin{align}
\label{eqn:nuc_rate}
\dot{n} &= (m - n(t))V_{box}k\\
\Leftrightarrow \quad \!\frac{\dot{n}}{m} &= \left(1 - \frac{n(t)}{m}\right) V_{box}k \nonumber\\
 &  \text{in the limit } m \rightarrow \infty \nonumber\\
\Leftrightarrow \frac{n(t)}{m} &= 1 - \exp\left( -V_{box} k t \right)\\
 &  \text{defining } \tau = (V_{box} k)^{-1} \nonumber\\
\label{eqn:nuc_rate_result}
\Leftrightarrow \frac{\dot{n}(t)}{m} &= \frac{1}{\tau} \exp\left( \frac{-t}{\tau} \right) 
\end{align}

The final result in \autoref{eqn:nuc_rate_result} is the well known stochastic exponential distribution. As the expectation value of the exponential distribution is given by its parameter $\tau$, the common approach of using the mean induction time when all simulations have nucleated yields an accurate result and precision can be obtained by taking a large number of simulations.

\subsection{Maximum likelihood estimator of the induction time}
\label{sec:ml_estimator}
In case the simulation time is not accessible we instead will have to deal with truncated exponential distributions. For this we can use maximum likelihood (ML) estimators. The derivation follows the one by Deemer and Votaw 1955\cite{Deemer1955}.\\

Maximum likelihood estimators are based on the idea that we can write down the expression of the total probability called likelihood $\mathcal{L}$ for a given set of measurements $x_i$ depending on parameters of the assumed underlying distribution. For the exponential distribution, parameterized by the characteristic decay rate $\kappa$, it is given by

\begin{align}
\label{eqn:exponential_product}
\mathcal{L}(\kappa) = \prod_{i=1}^N p(x_i) = \prod_{i=1}^N \kappa^N \exp\left( - \kappa x_i \right ) \; \text{.}
\end{align}

We continue by find the maximum of this product. To simplify it and also to evade overflow problems on floating point machines, the logarithm of the likelihood is used and maximized yielding the same parameters because the logarithm is a monotonic function and thus does not shift the extrema.\\
The maximum probability can then be found by usual means of analysis executed in \autoref{eqn:exponential_maximization} - \autoref{eqn:exponential_maximization_end}.

\begin{align}
\label{eqn:exponential_maximization}
& 0 \stackrel{!}{=} \left. \frac{\partial \log (\mathcal{L})}{\partial \kappa} \right|_{\kappa=\hat{\kappa}}\\
\Leftrightarrow \qquad  &0 = \left. \frac{\partial}{\partial \kappa} \left( N \log(\kappa) - \kappa \sum_{i=1}^N t_i \right)  \right|_{\kappa=\hat{\kappa}} \\
\Leftrightarrow \qquad &0 = \frac{N}{\hat{\kappa}} - \sum_{i=1}^N t_i \\
\label{eqn:exponential_maximization_end}
\Leftrightarrow  \qquad & \!\!\!\!\!\!\!\!\: \hat{\kappa}^{-1} = \frac{1}{N} \sum_{i=1}^N t_i  
\end{align}

By this we have found that the maximum likelihood estimator of $\kappa$, for a set of samples drawn from an exponential distribution, is given by the inverse arithmetic mean of the samples. This result is neither new nor surprising but is shown to illustrate how the method of maximum likelihood works. In the following we then show how to handle censored and truncated distributions by the maximum likelihood method.\\
Both distributions refer to sets of samples that are incomplete in the sense that they only include samples up to some threshold $t_i < T$, but while in the case of truncated distributions the number of samples larger than this threshold is unknown, for the censored distribution it is known. Taking the example of time consuming nucleation events in computer simulations we are in the case of censored distributions, as the total number of boxes is known but the simulation is just stopped at some point without all boxes having had a nucleation event. The probability of an event later than the censoring time $T$ is given by
\begin{equation}
\label{eqn:prob_t_larger_T}
p(t_i>T) = \int_T^{\infty} \kappa \exp(-\kappa t) dt = \exp(-\kappa t) \; \text{.}
\end{equation}
Therefore we can write the complete probability distribution as
\begin{align}
\label{eqn:pdf_censored}
f(t) = 
\begin{cases}
\kappa \exp(-\kappa t) & t < T\\
\exp(-\kappa T) & t \geq T\\ 
\end{cases} \; \text{.}
\end{align}


In the simulation we can then split up the number of boxes $N$, into $n$ boxes where a nucleation event was found, and $m = N -n$ others where no nucleation event was spotted during the simulation time $T$.\\
Further we have to account for the fact that the samples without distinct times are indistinguishable. This is done by weighting them with the number of possible permutations which are calculated in the binomial prefactor $\binom{N}{m}$. The whole expression for the likelihood function $\mathcal{L}(\kappa)$ then is given by \autoref{eqn:ml_censored_exponential} and the extremum of it is evaluated in the subsequent reformulations.

\begin{align}
\label{eqn:ml_censored_exponential} 
\mathcal{L}(\kappa) & = \left. \binom{N}{m} \;  \kappa^n \; \exp(- \kappa \sum_{i=1}^n t_i) \;  \exp(-\kappa T)^m \quad \right| \left.\frac{\partial \log ( ... )}{\partial \kappa} \right|_{\kappa=\hat{\kappa}} \\
\Leftrightarrow \quad\log ( \mathcal{L}(\kappa)) & = \left.\log\binom{N}{m}  + n \log ( \kappa) - \kappa \sum_{i=1}^n t_i - m \kappa T \quad \right| \left. \frac{\partial(...)}{\partial \kappa} \right|_{\kappa=\hat{\kappa}} \\
\Leftrightarrow \:\! \frac{\partial \log ( \mathcal{L}(\kappa))}{\partial \kappa} & = \left. \frac{n}{ \kappa} - \sum_{i=1}^n t_i - m  T \quad \right|_{\kappa=\hat{\kappa}}\\ 
 & \; \; \, \vrule
  \begin{aligned}[t]
      \raisebox{0.9cm}{ \makebox[0.1cm]{}} \raisebox{-0.5cm}{ \makebox[0.5cm]{}}  \text{with } \frac{\partial \log ( \mathcal{L}(\hat{\kappa})  )}{\partial \kappa}  \stackrel{!}{=} 0  \\
  \end{aligned} \nonumber\\
\Leftrightarrow \qquad\qquad \;\; \:\! 0 &= \frac{n}{ \hat{\kappa}} - \sum_{i=1}^n t_i - m  T \quad\\
\label{eqn:ml_censored_exponential_final}
 \Leftrightarrow \qquad\quad  \;\: \hat{\kappa}^{-1} &= \frac{1}{n} \left(  \sum_{i=1}^n t_i + m T \right)
\end{align}


The final line \autoref{eqn:ml_censored_exponential_final} is the estimator of the decay rate of the censored exponential distribution. It is used for the estimation of induction times to compare with other published results in the next sections.

\subsection{Monte Carlo uncertainty estimation}
\label{sec:mc_uncertainty}
Having determined the estimator for the nucleation rate, the next question concerns its uncertainty i.e what is the distribution of $\hat{\kappa}$? While corresponding literature on analytic expressions for the distribution has been published for example by Chen and Bhattacharyya 1988\cite{Chen1988}, the complexity becomes inappropriate for the task at hand. Thus we will follow instead a Monte Carlo approach described for example in the book Numerical Recipes\cite{Press1992} to find the uncertainty of the estimator.\\

For this purpose we draw samples from an exponential distribution characterized by the estimator calculated from the actual simulation data. Afterwards the samples are censored by cutting off all elements larger than $T$ and calculate the corresponding estimator $\hat{\kappa}_{MC}$ for the Monte Carlo sample. From multiple such random sets we can create a histogram of estimates for $\hat{\kappa}$ that can be seen together with some exemplary random samples in \autoref{fig:mc_example}. As the distribution seems to incorporate only little higher moments, the standard deviation of the distribution is used as the estimators uncertainty $\sigma_{\hat{\kappa}}$.\\

\begin{figure}[ht]
%\begin{center}
\subfloat[Exponentially distributed random samples of size 500 with an exemplary censoring time of $T=\kappa^{-1}$]{\includegraphics[width = 0.45 \textwidth]{mc_random_sample.pdf}} \hspace{0.5cm}
\subfloat[Distribution of $\hat{\kappa}$ for the previously generated MC samples. The distribution can be described mostly by mean and standard deviation as the number of estimates in the tails are small.]{\includegraphics[width = 0.45 \textwidth]{k_estimator_distribution.pdf}}  
\caption[Monte Carlo uncertainty estimation example]{Exemplary samples for a given $\kappa$ as well as the distribution of estimates calculated from the random samples. The uncertainty on $\hat{\kappa}$ is approximated by the standard deviation of the distribution from the corresponding Monte Carlo analysis at a given $\kappa$.}
\label{fig:mc_example}
%\end{center}
\end{figure}
Concerning the uncertainty in detail we can ask how long a simulation should last to yield precise results. For this we can first look at the case where $1 \gg \kappa T$ corresponding to a simulation where all boxes showed a nucleation event. In this case we have seen before that $\hat{\kappa}^{-1} = \frac{1}{N} \sum_{i=1}^N t_i$. As we assume that the $t_i$'s are exponentially distributed we know that $\sigma_{t} = \kappa^{-1}$. Gaussian error propagation then results in
\begin{align}
\label{eqn:uncertainty_k_gg}
\frac{\sigma_{\hat{\kappa}}}{\hat{\kappa}} = \frac{1}{\sqrt{N}} \; \text{.}
\end{align}
%\begin{align}
%\label{eqn:uncertainty_k_gg}
%\frac{\sigma_{\kappa}}{\kappa} &=\frac{1}{\kappa} \sqrt{\sum_{i=1}^N \left( \frac{\partial \kappa}%{\partial t_i} \right)^2 \sigma_{t_i}^2 }\\ 
%& \; \; \, \vrule
%  \begin{aligned}[t]
%  \raisebox{0.9cm}{ \makebox[1cm]{}} \text{with }  \frac{\partial \kappa}{\partial t_i} &= %\frac{\partial}{\partial t_i} \left( N \left( \sum_{i=1}^N t_i \right)^{-1} \right)\\
%  &= -N\left( \sum_{i=1}^N t_i \right)^{-2}  = \frac{\kappa^2}{N} \; \text{,} \\
% \raisebox{-0.4cm}{ \makebox[1cm]{}} \text{and } \sigma_{t} &= \kappa^{-1}
%  \end{aligned} \nonumber\\
% &= \frac{1}{\kappa} \sqrt{ N \left( \frac{\kappa^2}{N}\right)^{2} \kappa^{-2} } \\
% &= \frac{1}{\sqrt{N}}
%\end{align}
%\todo{Is das eine Tautologie!?}
Similarly we can take the limit of $1 \ll \kappa T$ which is the case when the mean nucleation time is much larger than the simulation time and therefore only a small fraction of the boxes hosted a nucleation event. In this case we can expand the estimator in the fraction of nucleated trajectories $\frac{n}{N}$ to find $\hat{\kappa} \approx \frac{n}{N} \frac{1}{T}$. In this case the decrease of nucleation events due to the smaller not nucleated volume is not seen yet and the only information about the nucleation rate is obtained from the number of boxes with nucleations compared to the total number of boxes. As $n$ is Poisson distributed we know that $\sigma_n = \sqrt{n}$. Fixing N and T and using the expectation value of nucleations $n = N \hat{\kappa} T$, the Gaussian error propagation for the relative uncertainty is given in \autoref{eqn:uncertainty_k_ll}.
%\begin{align}
%\label{eqn:uncertainty_k_ll}
%\frac{\sigma_{\kappa}}{\kappa} &= \frac{1}{\kappa} \frac{\sqrt{n}}{NT} \nonumber\\
%&=\frac{\sqrt{N \kappa T}}{N T \kappa} \nonumber\\
%&=\frac{1}{\sqrt{N \kappa T}}
%\end{align}
\begin{align}
\label{eqn:uncertainty_k_ll}
\frac{\sigma_{\hat{\kappa}}}{\hat{\kappa}} = \frac{1}{\hat{\kappa}} \frac{\sqrt{n}}{NT} =\frac{\sqrt{N \hat{\kappa} T}}{N T \hat{\kappa}} =\frac{1}{\sqrt{N \hat{\kappa} T}}
\end{align}
Finally we are also able to not only look at limits analytically, but also to approximate the relative uncertainty directly by means of the aforementioned Monte Carlo method. For this purpose the same procedure as before is used. The number of elements per sample is set consistently with the actual number of used simulations to 500 and to achieve good precision on the uncertainty, the standard deviation of 1000 samples is used. To compare the analytically derived limits of the uncertainty with the Monte Carlo results both are drawn into \autoref{fig:relative_uncertainty}.

\begin{figure}[ht]
\begin{center}
\includegraphics[width = 0.65 \textwidth]{rel_uncertainty_k.pdf}
\caption[Nucleation rate uncertainty depending on measurement time]{Relative uncertainty of the ML estimator for varying $\kappa T$. The x scale is chosen dimensionless such that it indicates the simulation time in comparison to the characteristic nucleation time.}
\label{fig:relative_uncertainty}
\end{center}
\end{figure}

We find that for the limits of $\kappa T \ll 1$ as well as $\kappa T \gg 1$, Monte Carlo and analytical results are in good accordance while in between the analytical limits only can be used as a rough estimate.\\

What can be seen from \autoref{fig:relative_uncertainty} is that the uncertainty of the estimation drops sharply until about half of the characteristic induction time, after which it only obtains little more precision. This is not surprising as the nucleation times contain the rate and more nucleation events occur at the beginning while long simulation times only add little further information. Thus simulating until all boxes hosted a nucleation event is only necessary if one wants to use the simpler arithmetic mean of the induction times as the setimator, or if any other constraints make it necessary to reach crystallization of all boxes.

\section{Nucleation rate comparison}
\label{sec:nucleation_rates}
Finally we are able to evaluate the induction time distribution to find the rates given in \autoref{fig:nucleation_comparison}.

\begin{figure}[h!]
\centering
\includegraphics[width=0.6 \linewidth]{nucleation_rate_comparison.pdf}
\caption[Nucleation rate comparison with literature values]{Experimental and theoretical examples of nucleation rates in the hard sphere system at different volume fractions from the literature \todo{citations and/or other data}, to compare with the own measured data points.}
\label{fig:nucleation_comparison}
\end{figure}

From the diagram we can state that our measurements confirm the previous simulation results, that still stand against the experimentally found ones. Further the results are calculated together with their statistical uncertainty which is mostly visible for the data point at $\eta = 53.1 \%$. It is also indicated for the others but due to the logarithmic scaling the uncertainty is almost not visible.\\
While uncertainties in the literature are often just very roughly given, the here presented method makes
it possible to quantify the statistical uncertainty of the rates and the large number of simulations give us high precision in comparison to rates elsewhere found. 

\section{Memory kernels of nucleating ensemble}
\label{sec:memory_kernels}
The approach by Hugues Meyer et al. 2019\cite{Meyer2019a}, to calculate memory kernels from an ensemble of trajectories, is used on the data discussed in the previous sections as well as on trajectories of a system characterized in \autoref{tab:system_16k_mem}. The second system is used because the first ensemble was neither simulated up to the point where most boxes contained a stable cluster nor until the boxes were fully crystallized as the transition width is large and takes very long simulations to fill up the large box. The second system's parameters are chosen to fulfill both objections.\\

\begin{table}[ht]
\centering
\begin{tabular}{c|c}
Parameter & Value \\ \hline
N & 16384 \\
eq\_steps/particle & 5000 \\
pr\_steps/particle & 200000 \\
$\eta_i$ & 45.0 \% \\
$\eta_f$ & 53.4 \% \\
\end{tabular}
\caption[Simulation parameters of data production system with 16384 particles]{Input parameters of simulations on the NEMO HPC cluster. The large number of production steps is chosen, together with the final volume fraction $\eta_f$, in a way to simulate nucleation and full crystallization of the boxs in almost all cases as can be seen in the top diagram of \autoref{fig:cluster_growth_example}. Furthermore the small box size leads to a small transition width $\Delta$ of about $150 \delta t$ corresponding closely to the width of the memory kernel, as lately shown by Meyer et al. 2021\cite{Meyer2021}.} 
\label{tab:system_16k_mem}
\end{table}

Still the memory kernel of the large system has been calculated but except of the Markovian contribution only little of the memory kernel was visibible, indicating that the sample is not sufficently long or that the largest cluster is not an appropriate observable for nucleations in large systems.\\

To compare the memory kernel with direct measurements of the observable the evolution of the ensemble is depicted in the top of \autoref{fig:cluster_growth_example}. The trajectories have been normalized by the number of particles in the box and some statistical properties likes percentiles and arithmetic mean are also shown as the large number of trajectories otherwise makes it hard to distinguish the actual density of lines at some points. At the bottom of the figure the share of trajectories at different stages of the nucleation process is identfied. For this it is assumed that trajectories below an normalized largest cluster of 0.1 can be identified as not nucleated, such trajectories above 0.5 as fully crystallized and all trajectories in between as in the crystallization process.\\

\begin{figure}[ht]
\centering
\includegraphics[width=0.7 \linewidth]{cluster_growth_quantities.pdf}
\caption[Largest cluster trajectories of small system with percentiles and average]{Top: Normalized trajectories of largest cluster with percentiles and arithmetic mean indicated. It can be observed that a fraction of the trajectories nucleates in more than one step where at first only about 60\% of the box is filled by the crystal and at later times they sometimes crystallize further until almost the complete box is filled by the solid phase. From \autoref{eqn:solid_fraction_result} we would expect an equilibrium solid fraction of 80\% by volume, closely corresponding to the expected solid fraction by particles.\\
Bottom: Fraction of trajectories within intervals chosen to identify nucleated trajectories, momentary growing trajectories and fully nucleated trajectories. As the growth process is much faster than the distribution of nucleations, the orange curve roughly resembles the derivative of the other two curves.}
\label{fig:cluster_growth_example}
\end{figure}

While for the large system only little of the crystallites reached the box boundaries in this latter we see that almost all clusters fill the whole box at the end of the simulation.\\
Further as there is no clear analysis yet on how the direct quantities and the memory kernel are related, the subdivision is an approach to show direct observables that possibly are related to properties of the memory kernel.\\

We see in on the left of \autoref{fig:memory_kernel}, that the shape of a memory kernel slice at some reference time is rather simple. For this reason we use a Gaussian fit to approximate the width and amplitude of the kernel. For this purpose we neglect the Markovian part of the kernel at around $t_1-t_2 \approx 0$. To validate the fit results we further use the FWHM, where the maximum is determined by the mean value of the peak's crest.\\
As the properly normalized results for both methods are in good agreement, we can conclude that the shape of the memory kernel in this case is mostly defined by a width and an amplitude over time which are depicted on the right of \autoref{fig:memory_kernel}.\\

\begin{figure}[ht]
\begin{center}
\subfloat[Slice through memory kernel at the given reference time. With the data the excluded Markovian part of the kernel is depicted. Further the full width at half maximum (FWHM) is shown as a first measure of the kernel width as well as a Gaussian fit. The FWHM is normalized to the value of a corresponding Gaussian curve.]{\includegraphics[width=0.37 \linewidth]{example_memory_kernel.pdf}} \hspace{0.5cm}
\subfloat[Top: Width of the memory kernel slices from FWHM and Gaussian fit. The FWHM is normalized to the value of a corresponding Gaussian curve.
Bottom: Amplitude of the memory kernel slice on the one hand by using the mean value of the data around the maximum and on the other hand by using the amplitude derived from the best Gaussian fit. The amplitude derived from the maximum value is normalized to the value of a corresponding Gaussian curve.]{\includegraphics[width=0.59 \linewidth]{memory_kernel_shape.pdf}}
\caption[Width and amplitude of memory kernel with example slice]{Example memory kernel together with width and amplitude depending on time.}
\label{fig:memory_kernel}
\end{center}
\end{figure}

The width of the memory kernel sections are mostly constant over the whole measurement with the exception that it becomes very noisy at the end.\\
The amplitude in comparison increases at the beginning, remains over a prolonged period of time constant and then declines towards the end of the measurement.\\
As published by Meyer et al. 2021\cite{Meyer2021}, the width of the memory kernel seems to depend on the phase transition time. Because for the hard sphere system the transition width is mostly given by the arbitrarily chosen box size, the dependence is possibly only an artifact with other memory effects buried beneath.\\ 

To seperate these memory effects one could generate trajectories with a purely Markovian approach, like Brownian dynamics, with corresponding characteristic properties. Then comparing the memory kernels of the purely Markovian ensemble with the a priori non Markovian hard sphere ensemble may help to distinguish memory effects due to the system size from those related to the dynamics of the fluid.\\

An other approach would be to use the committer probability of the largest cluster as an observable, as it would not include a direct system size dependence and by itself is already bounded between zero and one, which is a requirement for the memory kernel analysis.

\newpage

\chapter{Conclusion}
% !TEX root = writing_version.tex

%\section{Conclusion}
\label{sec:conclusion}

\begin{description}
\item[summary]
\item[key results]
\item[future outlook]
\end{description}

\newpage

\pagestyle{plain}
\renewcommand{\chapterpagestyle}{plain}
\chapter{Appendix}
% !TEX root = writing_version.tex

%\section{A}

\chapter{Derivation of momentum transfer in elastic hard sphere collisions}
\section*{}
\label{sec:appendix}
The elastic collision of two hard spheres $i$ and $j$ is governed by energy and momentum conservation.
In the collision of the two particles, the momentum transfer $\Delta \vec{p}$ is confined within the direction of relative position because the collision is instantaneous. Thus, we may write the momentum transfer on particle $i$ as $\Delta \vec{p}_i = a \vec{r}_{ij}$ with the relative position $\vec{r}_{ij} = \vec{r}_{j} -\vec{r}_{i} $ and $a$ being some constant to be determined. It can be easily seen that $\Delta \vec{p}_i = - \Delta \vec{p}_j$ due to $\vec{r}_{ij} = -\vec{r}_{ji}$. \\

Using $\vec{p}_i = m_i \vec{v}_i$, we can write the velocity after the collision for particle i as
\begin{align}
\vec{v}_i^{\,'} = \vec{v}_i +\frac{\Delta \vec{p}_i } {m_i} \; \text{.}
\end{align}
By using classical energy conservation and reformulating, we find
\begin{align}
\label{eqn:momentum_transfer}
\frac{1}{2}\left( m_i |\vec{v}_i|^2 + m_j |\vec{v}_j|^2 \right) &= \frac{1}{2}\left( m_i |\vec{v}_i^{\,'}|^2 + m_j |\vec{v}_j^{\,'}|^2 \right) \\
\Leftrightarrow \qquad  m_i |\vec{v}_i|^2 + m_j |\vec{v}_j|^2  &= m_i \left(  \vec{v}_i +\frac{\Delta \vec{p}_i } {m_i} \right)^2 + m_ j\left(  \vec{v}_j +\frac{\Delta \vec{p}_j } {m_j} \right)^2 \\
\Leftrightarrow \qquad\qquad\qquad\qquad\quad\! 0 &= 2 \left(  \vec{v}_i \cdot  \Delta \vec{p}_i  +\vec{v}_j \cdot  \Delta \vec{p}_j   \right)+\frac{|\Delta \vec{p}_i|^2 } {m_i} +\frac{|\Delta \vec{p}_j|^2 } {m_j}\\
\Leftrightarrow  \qquad\qquad\qquad\qquad\quad\! 0 &= 2 (\vec{v}_i - \vec{v}_j)\cdot \Delta \vec{p}_i + |\Delta \vec{p}_i|^2 \left(\frac{1}{m_i} +\frac{1}{m_j} \right)\\
 \Leftrightarrow \qquad\qquad\quad\, 2 a \, \vec{r}_{ij} \cdot \vec{v}_{ij} &= a^2 |\vec{r}_{ij}|^2 \frac{m_i + m_j}{m_i m_j}\\
 \Leftrightarrow\qquad\qquad\qquad\qquad\;\;\; a &= \frac{2 m_i m_j }{(m_i + m_j)  } \frac{\vec{r}_{ij} \cdot \vec{v}_{ij}}{|\vec{r}_{ij}|^2}\\
\label{eqn:momentum_transefer_result}
\Rightarrow \vec{v}_{i/j}^{\,'} &= \vec{v}_{i/j} \pm \frac{2 m_{j/i} }{(m_i + m_j)  } \frac{\vec{r}_{ij} \cdot \vec{v}_{ij}}{|\vec{r}_{ij}|^2}
\end{align}

Equation \ref{eqn:momentum_transefer_result}, is the final result which is used in the thesis.

\newpage

\chapter*{Acknowledgements}% for the actuall unnumbered heading

I thankfully want to achknowledge the freedom
The authors acknowledge support by the state of Baden-Württemberg through bwHPC
and the German Research Foundation (DFG) through grant no INST 39/963-1 FUGG (bwForCluster NEMO).

%Phillips
%It is a pleasure to acknowledge the fruitful discussion and helpful comments from
%Tanja Schilling, Anja Kuhnhold, Andreas Härtel and Graziano Am-
%ati. Especially, I am grateful for the support from Hugues Meyer. He provided
%additional simulation trajectories from the HPC facility of the University of Lux-
%embourg and gave me permission to use his memory-kernel reconstruction scheme
%that is not published yet. He also invested a lot of time to present and discuss
%his scheme for clarifications. The simulation data presented in this work was
%supported by the state of Baden-Württemberg through bwHPC and the German
%Research Foundation (DFG) through grant no INST 39/963-1 FUGG.



\printbibliography

\end{document}







